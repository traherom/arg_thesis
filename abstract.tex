\par This thesis explores the viability of using \ac{IP} address hopping in front of a network as a defensive measure. Network address space randomization techniques theoretically provide protection to a network by appearing to randomly change the addresses of hosts inside, presenting a challenge to an intruder attempting to break in and map the network. This research presents a custom gateway-based \ac{IP} hopping solution called \ac{ARG} that combines previous work in this area.

\par \ac{ARG} works as a transparent gateway in front of a network, requiring no changes to the hosts inside or out. Each \ac{ARG} gateway is configured with a small amount of knowledge on one or more other gateways, allowing them to connect and configure themselves to pass fully encrypted and authenticated traffic between themselves. Connections to non-\ac{ARG} networks or hosts are handled gracefully, allowing long lived connections to exist without terminating them during \ac{IP} address changes. This thesis tests the overall stability of \ac{ARG}, the accuracy of its classifications, the maximum throughput it can handle, and the maximum rate at which it can change \acp{IP} and still communication reliably.

\par \tbd{Tests demonstrate...}


