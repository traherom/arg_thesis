\acresetall

\par This thesis explores the viability of using \ac{IP} address hopping in front of a network as a defensive measure. Network address space randomization techniques theoretically provide protection to a network by appearing to randomly change the addresses of hosts inside, presenting a challenge to an intruder attempting to break in and map the network. This research presents a custom gateway-based \ac{IP} hopping solution called \ac{ARG} that combines previous work in this area.

\par \ac{ARG} works as a transparent gateway in front of a network, requiring no changes to the hosts inside or out. Each \ac{ARG} gateway is configured with a small amount of knowledge on one or more other gateways, allowing them to connect and pass fully encrypted and authenticated traffic amongst themselves. Connections to non-\ac{ARG} networks or hosts are handled gracefully, allowing long-lived connections to exist without terminating them during \ac{IP} address changes. This thesis tests the overall stability of \ac{ARG}, the accuracy of its classifications, the maximum throughput it can support, and the maximum rate at which it can change \acp{IP} and still communicate reliably.

\par This research is accomplished on a physical test network with nodes representing the types of hosts found on a typical, corporate-style network. Direct measurement is used to obtain all results for each factor level. Tests demonstrate \ac{ARG} classifies traffic correctly, with no false negatives and less than a 0.15\% false positive rate on average. The test environment conservatively shows this to be true as long as the \ac{IP} address change interval exceeds two times the network's round-trip latency; real-world deployments may allow for more frequent hopping. Results show \ac{ARG} capably handles traffic of at least four megabits per second with no impact on packet loss. Fuzz testing validates the stability of \ac{ARG} itself, although additional packet loss of around 23\% appears when under attack.

\acresetall

