\documentclass[]{afit-etd}

% The afit-etd class requires the following packages: url, refcount, graphicx,
%                                                     sf298, hyperref
%
% Required files to support the afit-etd class are:
%      afit-etd.cls
%      afitlogo.pdf or afitlogo.eps
%      af298.sty (slight modifications required to fix a 'glitch')
% All of the required files must be located in your LaTeX search path.  The
% easiest place to put them is in the working directory along side with your
% thesis.tex file.
%
% Additional files used in this shell but not required are:
%     thesis.bib (used as an example only)
%     thesnum3.bst (can be replaced with any other bibliography style file)
%     CampusPhoto.pdf and CampusPhoto.eps (used as an example only)
% This shell will not process without these files, but if you delete sample
% text and replace the BST file with another, then these will not be required
% at all.


% The following packages and macros are recommended but not required:
\usepackage{xcolor}                  % standard package for including colors
\usepackage[square,sort&compress,numbers]{natbib} % better citations
                                % especially when including a string of
                                % citations
%\newcommand\citenum[1]{#1}
\usepackage{subfig}

% Packages for fonts
\usepackage{amssymb}  % math symbols (amsmath & amsthm are defined in afit-etd)
\usepackage[printonlyused]{acronym}
\usepackage{lettrine} % provides dropped characters to lead off a paragraph.
                      % These are not required, but some people like them at
                      % the start of each chapter.
\usepackage{bm}       % The most comprehensive package for bold math figures
\usepackage{textcomp} % \texttimes, \textdegree, \textohm, \textmu

% Packages for tables
\usepackage{booktabs} % improved rules (lines) for tables
\usepackage{dcolumn}  % align at decimal in tables
\usepackage{multirow} % table elements spanning multiple rows

\graphicspath{{figures/}} % reduce clutter by storing figures elsewhere

% Defining macros for all references makes it easy to change format later from
% "Equation 1" to "Eqn (1)", for example
\newcommand{\fig}[1]{Figure~\ref{fig:#1}}
\newcommand{\tab}[1]{Table~\ref{tab:#1}}
\newcommand{\eq}[1]{Equation~\eqref{eq:#1}}
\newcommand{\eqtwo}[2]{Equations~\eqref{eq:#1} and \eqref{eq:#2}}
\newcommand{\chap}[1]{Chapter~\ref{chap:#1}}
\newcommand{\sect}[1]{Section~\ref{sec:#1}}
\newcommand{\Ref}[1]{Reference~\citenum{#1}}

% Other stuff I want
\usepackage{placeins}
\usepackage{comment}

% Todo tag
\newcommand{\tbd}[1]{{\color{red} TBD: #1}}

%%%%% Required front matter definitions %%%%%%%%%%%%%%%%%%%%%%%%%%%%%%%%%%%%%%%

\title  {Malicious Traffic Detection through Internet Protocol Address Hopping}
\doctype{THESIS} % or GRADUATE RESEARCH PAPER, DISSERTATION, or REPORT
                 % REPORT will generate a simplified format more suitable for
                 % class assignments

\author          {Ryan A.}{Morehart}
\rank            {Second Lieutenant, USAF}
\previousdegrees {B.S.} % Abbreviate any previous degrees

% Uncomment the following lines if there is a second author
% \coauthor          {FirstName I. LastName} 
% \corank            {Major, USAF}
% \copreviousdegrees {B.S.} 

\degree          {Master of Science}
\graduation      {27}{March}{2013} % format is {DD}{Month}{YYYY} where
                                   % Month must be: March, June,
                                   % September, or December

\designator{AFIT/GE/ENG/12-XX} % assigned by the graduate advisor in
                               % during the student's final quarter


\distribution{DISTRIBUTION STATEMENT A:\\APPROVED FOR PUBLIC RELEASE;
  DISTRIBUTION UNLIMITED} % or other appropriate distribution statement from the
                          % AFIT Style Guide

\committee{ % Advisor must be listed first in the list of committee members
  {Dr. Barry Mullins, PhD (Chairman)},
  {Dr. Rusty Baldwin, PhD (Member)},
  {Dr. Timothy Lacey, PhD (Member)}
}

\department {Department of Electrical and Computer Engineering}
\deptsymbol {ENG}

\school     {Graduate School of Engineering and Management}
\dean       {M. U. Thomas} % only used for PhD dissertations

% Uncomment the following line to switch from blank signature lines on
% the approval page to lines marked with ``/signed/'' and the
% corresponding dates.  This avoid having to scan the signature page into the
% final PDF document for the electronic version, but it also doesn't look as 
% professional.  Similarly, the Dean's signature can be indicated using the 
% second line below.  Note that the original signatures are still required on
% the hardcopy submitted to the library.

% \committeeSignedDates{2 Jun 2011,3 Jun 2011,4 Jun 2011}
% \deanSignedDate{9 Jun 2011}

\abstract{Insert abstract here.}
% The abstract can also be saved in a separate file, called abstract.tex for 
% example, and included here using: "\abstract{\par This thesis explores the viability of using \ac{IP} address hopping in front of a network as a defensive measure. Network address space randomization techniques theoretically provide protection to a network by appearing to randomly change the addresses of hosts inside, presenting a challenge to an intruder attempting to break in and map the network. This research presents a custom gateway-based \ac{IP} hopping solution called \ac{ARG} that combines previous work in this area.

\par \ac{ARG} works as a transparent gateway in front of a network, requiring no changes to the hosts inside or out. Each \ac{ARG} gateway is configured with a small amount of knowledge on one or more other gateways, allowing them to connect and pass fully encrypted and authenticated traffic amongst themselves. Connections to non-\ac{ARG} networks or hosts are handled gracefully, allowing long-lived connections to exist without terminating them during \ac{IP} address changes. This thesis tests the overall stability of \ac{ARG}, the accuracy of its classifications, the maximum throughput it can handle, and the maximum rate at which it can change \acp{IP} and still communication reliably.

\par Tests demonstrate \ac{ARG} classifies traffic correctly, with no false negatives and less than a 0.15\% false positive rate on average. This remains true as long as the time between \ac{IP} address changes exceeds four times the one-way latency on the network, although test conditions give reason to suspect the hop rate could be closer to the latency in real operation. Tests show \ac{ARG} capably handles traffic of at least four megabits per second with no impact on packet loss. Fuzz testing validates the stability of \ac{ARG} itself, although additional packet loss appears when under attack.

\acresetall

}"

%%%% Required SF298 macros %%%%%%%%%%%%%%%%%%%%%%%%%%%%%%%%%%%%%%%%%%%%%%%%%%%

\DatesCovered{Oct 2011--Mar 2013} % First quarter of classes to Graduation
\ContractNumber{}   % "in house" if AFIT sponsored or blank otherwise
\ProjectNumber{}    % JON number (per advisor) or blank
\SponsoringAgency{} % sponsor address or '\relax' (will appear blank)
\Acronyms{}         % sponsor unit/office symbol or blank
\SMReportNumber{}   % blank unless sponsoring agency assigned a report number
\AddlSupplementaryNotes{}   % Add any other comments as necessary
\ReportClassification {U}   % document classification
\AbstractClassification {U} % abstract classification
\PageClassification {U}     % SF 298 classification
\AbstractLimitation {UU}    % change to 'SAR' if limited distribution
\SubjectTerms{Networks,}
\RPTelephone {(937) 255-3636 ext. XXXX}  % advisors 4 digit extension

%%%% Optional macro definitions %%%%%%%%%%%%%%%%%%%%%%%%%%%%%%%%%%%%%%%%%%%%%%%

%\dedication{\centering Insert optional dedication or remove/comment out this line completely.  The text will be centered vertically and horizontally on the page} 

%\acknowledgments{Insert optional acknowledgments or remove/comment out this line completely.} 
% If you prefer to provide "acknowledgements" instead (note the added "e"
% between the "g" and the "m") then add the "e" in the macro name so that
% it reads "\acknowledgements".}

%\vita{Insert optional vita or remove/comment out this line completely.}

% The default disclaimer and copyright statement is included by default.  An
% alternate disclaimer for foreign students or others can also be
% used by uncommenting the following line:
%
% \govtdisclaimer{Alternate Disclaimer.//See the Style Guide for more information}    

% The List of Tables and Figures can be omitted if not needed:
%
% \notables  
% \nofigures

% Additional "lists" can be added to the end of the front matter using the
% \addlistof macro.  For example, one might insert a list of symbols using the
%  tabbing environment with:
\addlistof{Symbols}{
\begin{tabbing}
  Symbol\quad \= Definition \kill % This sets up the tab stop
  Symbol \> Definition \\
  $I$    \> current (amps) \\
  $R$    \> resistance (ohms) \\
  $V$    \> voltage (volts) \\
  \null\\
  \textit{Subscripts}\\
  $rms$ \> root mean square \\
  $t$ \> true \\
  $0$ \> nominal\\
  \null\\
  \textit{Superscripts}\\
  $b$ \> body reference frame \\
  $n$ \> navigation reference frame \\
\end{tabbing}
}

% Alternatives to the preceding list of symbols and the following list of
% acronyms can be created using:
%
% \listofsymbols
% \listofabbreviations[5em]
% 
%% where the corresponding symbols and abbreviations must then be marked at
%% their first occurence in the text with "\addsymbol{Definition}{Symbol}" or
%% "\addabbrev{Definition}{Symbol}", respectively.  If the symbols are
%% too wide for the table, the alloted width can be increased by including
%% an optional width in square brackets, as in  "\listofsymbols[.3in]".  
%% Both of these lists will be listed in the order that they appear in the text.

\begin{document}

% The acronym package can be used to add a list of acronyms.  Because
% the acronym formatting is modified to match the AFIT Style Guide,
% the following command must be in the body of the document prior to the
% call to \makePrefatoryPages
\listofacronyms{
	\begin{acronym}[WPAFB]
	\acro{ARG}{Address Routing Gateway}
	\acro{IP}{Internet Protocol}
	\acro{NASR}{Network Address Space Randomization}
	\acro{IPv4}{IP version 4}
	\acro{IPv6}{IP version 6}
	\acro{TOTP}{Time-Based One-Time Password}
	\acro{ISP}{Internet Service Provider}
	\acro{NAT}{Network Address Translation}
	\acro{ARP}{Address Resolution Protocol}
	\acro{MAC}{Media Access Control}
	\acro{TCP}{Transmission Control Protocol}
	\acro{SHA}{Secure Hash Algorithm}
	\acro{HMAC}{Hashed Message Authentication Code}
	\acro{AES}{Advanced Encryption Standard}
	\acro{CPU}{Central Processing Unit}
	\acro{IPsec}{IP Security}
	\acro{RSA}{Rivest, Shamir, and Adleman}
	\acro{HOTP}{HMAC-Based One-Time Password}
	\acro{DYNAT}{Dynamic Network Address Translation}
	\acro{HTTP}{Hypertext Transport Protocol}
	\acro{VPN}{Virtual Private Network}
	\acro{APOD}{Applications that Participate in their Own Defense}
	\acro{COTS}{Commercial Off-The-Shelf}
	\acro{DARPA}{Defense Advanced Research Projects Agency}
	\acro{DHCP}{Dynamic Host Configuration Protocol}
	\acro{NAH}{Network Address Hopping}
	\acro{TAO}{Transparent Address Obfuscation}
	\acro{UDP}{User Datagram Protocol}
	\acro{CUT}{Component Under Test}
	\acro{SUT}{System Under Test}
	\acro{Mbps}{megabits per second}
	\acro{Kbps}{kilobits per second}
	\acro{ICMP}{Internet Control Message Protocol}
	\acro{VLAN}{Virtual Local Area Network}
	\acro{PCAP}{Packet Capture}
	\acro{pps}{packets per second}
	\acro{RAM}{Random Access Memory}
	\acro{CI}{confidence interval}
	\acro{DOS}{Denial of Service}
	\acro{RTT}{Round-Trip Time}
	\acro{IDS}{Intrusion Detection System}
	\acro{IV}{Initialization Vector}
\end{acronym}


}

% The following line is required to generate the prefatory pages
\makePrefatoryPages 

%% Body of the text follows, using \chapter, \section, \subsection,
%% \subsubsection, \paragraph, and \subparagraph to generate the
%% section headings.  For convenience, it may be useful to break the
%% full document into separate files, perhaps divided by chapters.  In
%% that case, the files would be loaded here using "\input{filename}"

\chapter{Introduction}
\lettrine{T}{his} is where you describe the problem you are trying to solve, describe what others
have done with this problem, and tell where your work fits into all of this.
- This chapter will typically include the following topics (not necessarily in this order):

\par Traditional network defenses consist of largely static tools; firewalls and intrusion detection systems (IDS) form the backbone of protection in most information technology shops.  Recently, however, there has been an interest in more active defense mechanisms, such as reputation and trust-based security \cite{Untrustworthiness} and network address space randomization (NASR) \cite{APOD, NAH}. This paper focuses on the latter in a military setting.

\par At a high level, the concept of NASR is simple: rather than a system sitting on a single Internet Protocol (IP) address, it changes its address rapidly, hopping amongst a set of IP addresses assigned to it. This is similar in concept to a frequency hopping radio, although the effect varies. 

\par Normally an attacker wishing to target a given network is capable of gaining a great deal of intelligence through simple scanning, checking each IP inside the network and then checking each port on the active IPs to see what services are available. With this knowledge, the attacker can almost certainly find an entrance into the network. IP hopping mitigates this issue by making it difficult to probe the network in the first place and quickly invalidating any network map that the attacker does manage to generate; even if they do manage to look a system's internal IP at one point in time, the address will change just moments later.

\par In this paper we propose an IP address hopping system called the Address Routing Gateway (ARG). It incorporates many of the features of previous address-hopping schemes, with an eye on the specific needs of the military. In this context each of the existing systems presents drawbacks that we attempt to avoid with ARG. Additionally, the design of ARG is intended to allow its future integration with a traditional IDS and honeypot, potentially gaining additional insight into an attacker's behavior.


\cite{rfc2}

\chapter{Background}
\lettrine{I}{n} this chapter, you describe all of the things that the reader would need to know in
order to understand the rest of the thesis. Save anything that is unique to your work
for Chapter 3. Chapter 2 should only include information that is general in nature.
For example, if you are designing a new routing protocol, you would describe routing
protocols in general in Chapter 2, and leave your actual protocol design for Chapter 3.
One of the main purposes for Chapter 2 is to make it easier to write Chapter 3,
because when you get to the various topics already covered in Chapter 2, you can just
reference back to that chapter, rather than have to describe the basic theory along with
whatever you uniquely did.

\subsection{IP Address Hopping in Detail}
\par As the introduction mentions, address hopping is a simple concept at a high level: we take the basic identifiers of a network and mutate them in a way that only those we wish to communicate with can follow. In doing so, we make it difficult for an adversary to correlate sniffed traffic with individual machines and even more difficult to probe into the network to enumerate hosts. In trying to actually implement such a system, however, several issues arise. 

\par To aid the discussion of these problems we will use the example network illustrated in Figure \ref{fig:exnetwork}. As it shows, we have two main networks \textit{A} and \textit{B} that are assigned the displayed IP ranges, are connected by the Internet, and have an interest in communicating freely with one-another. Each of these has a few friendly ends nodes (\textit{A1}, \textit{A2}, etc) behind a main router (\textit{AR} and \textit{BR}). Additionally, network \textit{B} has a potentially rogue client inside it named M1. Outside of those two networks, we see the friendly \textit{C2} node, who has an interest in at least occasionally communicating with nodes inside \textit{A}/\textit{B}, and malicious \textit{M2}, who wants access to said networks. The details of the routes between them and \textit{A} and \textit{B} are inconsequential.

\par Note that for the sake of this discussion non-routable IPv4 addresses are largely used. This is done merely for convenience and readability, the discussions apply to IPv6 as well unless otherwise noted.

% TBD modify image to include IPs?
\begin{figure}
	\centering
	\includegraphics[width=0.75\textwidth]{../diagrams/exnetwork}
	\caption{Example network layout}
	\label{fig:exnetwork}
\end{figure}

\par There are two basic ways IP address hopping could be deployed on this network: each end point hops individually or the network gateways transform incoming and outgoing packets. Both options and their strengths and weaknesses are discussed below.

\subsubsection{End Point Hopping}
\par For our example network, each end point hopping would mean that all of the nodes behind \textit{AR} and \textit{BR} (\textit{A1}, \textit{A2}, \textit{B1}, etc.) change addresses on a periodic basis, independent of one another. Despite the apparent simplicity of this setup, several questions must be answered.

\par First, how do the nodes keep track of one another? If every node knows about all the others, then scalability might become an issue, as every client presumably has to maintain some amount of data on fellow hopping clients to determine where each one is at any given time. It may be possible to devise a scheme where this flaw is mitigated by having all clients hop using the same secret and they each know just the broad IP ranges where fellow hoppers reside (i.e., the \textit{AR} nodes know that 10.2.2.0/24 is a network they talk to with hopping), but this accentuates the next question: how do clients choose their hops?

\par Clearly, each node must know some secret from which IP addresses are generated, hopefully in a manner which appears random to outsiders yet is predictable for those with the secret (the ``hopping pattern''). There may be only one of these secrets, which every node in the protected network knows---that is, the nodes in both A and B, hereafter collectively referred to as the ``hopping network.'' This does have the potential flaw of revealing too much information to an eavesdropper though: with a large number of nodes all following a hopping pattern based on the same key, it may be easier to deduce the secret. If one assumes that end-point hopping would \textit{have} to employ a single key to remain scalable, then the first flaw in this setup becomes obvious.

\par Just as importantly, however, is the question of how nodes coordinate their IP address choices. IP routing requires addresses follow a largely hierarchical setup, with broad ranges like 10.0.0.0 leading to 10.1.0.0 to 10.1.1.0 and so on. Thus, if \textit{A} has the network address range 10.1.1.0/24, then all of its nodes should fall within the address range of 10.1.1.1 through 10.1.1.255. This means that when each node hops it must remain within the valid range of the network it is a part of \textit{and} that the address it chooses must not already be taken another node. Given enough nodes in a subnet, a conflict is quite possible, leading to unpredictable network behavior. 

\par It is likely possible to devise a reasonably robust algorithm for all nodes to hop simultaneously and end up on unique IPs, but this encounters a few issues with the link-layer of a network's architecture. On local networks, machines are identified by a Media Access Code (MAC) and systems must map IP addresses to the hardware MAC using Address Routing Protocol (ARP). Nodes virtually always cache this information for speed, so for the period of time from a hop to the cache entry expiration time, packets sent to nodes on the local network will likely contain an incorrect MAC and not arrive at the correct destination. TCP would then detect the packet loss and retransmit, trying until either the connection timed out or the ARP cache updated. Even if ARP information is only cached for a few seconds, at best this could result in bursts of traffic on local networks and at worst dropped connections. A similar problem also exists at the switches in the network, although this would likely resolve itself quickly.

\par It would be possible to work around this issue by having whatever utility is handling the hopping also alter the ARP cache. To do so, every node would need to compute the correct locations of all other nodes and do so at precisely the same time. While not infeasible, this becomes even more platform-specific than the general IP hopping was. Approaching it the other way, nodes could all send gratuitous ARPs when their IP changes, but this still generates extra traffic on the network and \textit{forces} systems to be extremely vulnerable to ARP spoofing, a common part of a network attack.

\par The easiest solution to this problem is to give each node a unique, non-overlapping address range in which to hop. This avoids the ARP problem because there will never be a different node that used to have the same IP but a different MAC. However, this has the difficulty of requiring a large enough address space to make hopping beneficial. If a given node only has five possible addresses in which to hop, for instance, it becomes trivial for an attacker to just keep trying a single one until the node returns to it. IPv6 would avoid this problem, as the Internet Engineering Task Force recommends the allocation of a /64 address space ($2^{64}$ addresses) to every link \cite{rfc3267}, but IPv4 with its extremely limited address space would not allow this flexibility and, unfortunately, the reality of current networks mandates support for IPv4. % TBD see if a paper has an analysis of the needed address space for usability

\par Despite these negatives end point hopping does have advantages. First, scalability issues lie more in storage space and key lookup than actual computation, as every node only has to perform packet transformations for their own ingress and egress packets. Although not discussed yet, it is likely that any IP hopping scheme will also incorporate packet encryption, so distributing this load certainly cannot hurt. Second, end point hopping protects clients from probing no matter where the adversary is in the network. For example, as long as \textit{M1} in our example network lacks the hopping key, they have no more of an advantage in scanning any of the \textit{A} or \textit{B} nodes than \textit{M2}, who is outside the network. Finally, end point hopping comes with the ability for individual nodes such as \textit{C1} to connect to the main hopping network, without requiring any additional work or software development.

\subsubsection{Gateway hopping}
\par The alternative to end point hopping is to move the hopping to network gateways. In such a scheme, networks that we wish to protect are placed behind gateways that alter all traffic passing through them appropriately. What ``appropriately'' means varies with every implementation, but in one way or another, a gateway outside of the actual end points alters the IP traffic to make it appear as though the systems inside are changing IP addresses.

\par Nodes inside the network may or may not have knowledge of the hopping. In most instances the hopping occurs with no modification of the end points and is largely transparent. This need to only deploy a small number of systems, rather than altering every system, gives gateway-based hopping an advantage over end point-based on larger networks. Applying software and/or hardware changes to every system is costly in terms of both time and manpower. Even more significantly, legacy systems running older operating systems would likely need custom solutions and might be completely unsupportable as a result.

\par The most common observable side effect of gateway hopping (beyond the latency associated with the additional processing) is TCP connection dropping. Because TCP depends on IP addresses and ports numbers to identify on-going connections, any alterations to this information would traditionally kill the connection. This is a problem also faced by end point hopping schemes, but is more easily corrected because the individual machines know the state of connections and can correct appropriately. With gateway hopping, however, the situation can be much more delicate and likely requires significant state tracking at the gateway.

\par Because of this required state tracking and the need for a single system to alter all traffic in and out of an entire network, gateway-based hopping presents a possible performance problem. This should not be an insurmountable obstacle, however, and some studies have already shown a CPU impact of around 10\% \cite{TAO} (even when encryption was being applied). Additionally, gateway hopping has difficulty with individual nodes connecting to the network. In \textit{C1}'s case, for instance, it would likely need software for just the individual machine, almost exactly as an end-point hopping scheme would require.

\subsection{Other stuff to cover}
\begin{enumerate}
\item IPv6 packet structure?
\item TOTP
\item a
\end{enumerate}

\subsection{Previous Implementations}
\par With the proceeding discussion in mind, let us look at some previous implementations of this concept and the important results from their experimentation.

\subsubsection{BBN's Dynamic Network Address Translation (DYNAT)}
\par In 2001, BBN Technologies released a paper entitled ``Dynamic Approaches to Thwart Adversary Intelligence Gathering'' \cite{BBNDYNAT}. In this paper, they set out to test the hypothesis that ``dynamic modification of defense structure improves system assurance.''

\par Their dynamic network address translation (DYNAT) technique, as they called it, was put through a series of red-team experiments to test if it decreased an adversary's ability to map the network. The experimentation confirmed BBN's hypothesis: DYNAT did in fact increase system assurance because the adversary's work greatly increased compared to static networks. Even when the red team was given intimate knowledge of DYNAT's operation, the adversary could not identify a critical server in an enclave with DYNAT active.

\par The BBN's DYNAT implementation focused on individual clients connecting to a server enclave, through a DYNAT gateway on the server end. This gateway transformed incoming and outgoing packets between ``true'' host identification information---e.g., the actual IP address and port number of a server inside the enclave's network---and values which varied based on a pre-shared key and time. On the client side, a ``DYNAT shim'' sat in the network stack and did the same thing, transparently allowing client applications to work with the server enclave. Additionally, DYNAT applied encryption to all traffic for confidentiality.

\par BBN's experiments also demonstrated that the encryption of the packets was critical, as the attackers could trivially sniff the traffic to find important servers, even if they did not know the real IP address or port of the target. For example, an attacker could see a packet contained a HTTP response and thus learn an active IP and port for a web server, even if probing for it was impossible. While this information would only be valid for a limited period, it may be enough time for the attacker to compromise the internal network.

\subsubsection{Sandia Dynat}
\par In 2002, Sandia National Labs released a final report on their extensive work in the ``dynat'' field \cite{SandiaDynat}, as they refer to it. This report covers virtually every variable in a dynat system, from how hopping is synchronized to where in the network it is implemented. This paper points to many of the important issues that must be considered when implementing or deploying a dynat.

\par Of particular importance to our implementation proposal (ARG) are Sandia's recommendations on the location of the deployment of a gateway-based dynat. In order to avoid interference with existing firewall rules---particularly ones with a stateful firewall---, a dynat must be deployed beyond the current system. Likewise, for gateway-based virtual private networks (VPN), there is often a static IP requirement to allow for authentication \cite{SandiaDynat}, so a IP hopping gateway must also lie beyond the VPN concentrator. Essentially, the hopping gateway should be the last system before each the network connects to the outside world \cite{SandiaDynat}.

\par The Sandia report also provides significant insight into the interaction of a dynat with IPSec and strongly suggests a combination of the two. First, the encryption from IPSec avoids the ineffectiveness of dynat if the packets can be trivially sniffed for information, as already discussed in \cite{BBNDYNAT}. Second, IPSec is strengthened with the addition of a dynat, as the dynat can quickly reject invalid packets based on invalid source and destination identifiers, rather than forcing IPSec to perform expensive HMAC computations and/or encryption. However, the report also warns that the use of IPSec with dynat can reduce some aspects of dynat's access control because more identifiers are encrypted and unusable.

\subsubsection{Applications Participating in their Own Defense (APOD)}
\par In 2003, BBN proposed another IP hopping implementation as part of the Defense Advanced Research Projects Agency (DARPA) Applications that Participate in their Own Defense (APOD) project \cite{APOD}. This system was a refined version of the previous BBN DYNAT, featuring a network address translation (NAT) gateway sitting either on the server host itself or on a gateway into the network.

\par As noted by the authors, the primary differences between APOD and the previous DYNAT related to implementation. Whereas the BBN DYNAT was a very specialized solution, APOD employed standard Commercial Off The Shelf (COTS) utilities, such as Linux's iptables, to perform much of its work. They furthermore noted that APOD could be implemented as a NAT gateway on the networks of both the client and the server, the approach chosen for ARG.

\subsubsection{Network Address Space Randomization (NASR)}
\par The 2005 Network Address Space Randomization (NASR) was an IP address hopping system designed to defeat hitlist-based worms \cite{NASR}. These worms spread to pre-collected lists of IP addresses and typically propagate much faster than traditional worms that target random IPs. To fight this, NASR caused the pre-built hitlists to decay by changing IP addresses on a periodic basis.

\par The most unique aspect of this research was the use of Dynamic Host Configuration Protocol (DHCP) to force the changes. Through the use of a slightly intelligent DHCP server that leased IPs for a only a short time frame (on the order of tens of minutes) and only offered IPs that have not been used recently, most networks already using DHCP can be quickly changed to a randomized scheme. This simplicity does come at a cost, however: TCP connections are killed whenever the IP change occurs, forcing the hopping period to be quite long or risk unacceptable connection losses. The researchers did introduce intelligence into the DHCP server to allow it to detect ``long-lived'' TCP connections (i.e., a download) and give clients the same IP if they appeared busy. Beyond that, they also monitored what services a client was using, as many are resilient to a connection being torn down \cite{NASR}.

\par Despite those improvements, address changes occurred even in the fastest of instances only once an hour or so. This met the goal of hitlist worm protection, but is likely inadequate for obfuscating the network from a more intelligent enemy.

\subsubsection{Network Address Hopping (NAH)}
\par In 2005, European researchers presented a system they named Network Address Hopping (NAH) \cite{NAH}. This system focused on a client contacting a server as a negotiable protection measure, rather than an always-on system used between pre-configured systems.

\par A protocol employing IPv6 allowed a client to tell a server that they supported (and wished to use) NAH. If the server supported NAH, it replied with its hopping pattern. The client then sent its own hopping pattern, before reconnecting using the pattern the server just gave it. Packet count per connection was used to synchronize the hops and detect lost packets.

\par Once again, the NAH authors noted that encryption was important to maintaining the confidentiality of the data stream. However, they also stated that without encryption the system still provided some benefit, as packets bound for ``different'' addresses might follow differing routes due to the different (perceived) destinations. This means an attacker would need to either compromise a route fairly close to an endpoint to ensure they saw all traffic or compromise every possible route and collate the traffic together. Even if they manage to accomplish that, the attacker would still have to collect all traffic passing them in order to reconstruct the full stream (because they are unable to filter for specific IPs to identify the connection they are interested in), which poses a storage and computation problem given enough data \cite{NAH}.

\par As an additional side effect of the variable routing, the researchers noted that such a system may actually increase the throughput and reliability of a system. If multiple routing paths are used, congestion may be avoided and traffic ultimately flows more smoothly \cite{MultimediaDistributed}. While this is not viewed as an important aspect of this system, the potential does add support to the employment of address hopping.

\subsubsection{Transparent Address Obfuscation}
\par Finally, in 2006 a system called Transparent Address Obfuscation (TAO) was proposed \cite{TAO}. This system focused on protection of the Internet as a whole from hitlist-based worms and was somewhat based on the previous work in NASR \cite{NASR}. It featured gateways on networks that maintained external-to-internal address mappings for all nodes inside the protected network, with the external addresses changing with a configurable frequency. To maintain existing connection regardless of mapping changes, TAO included a NAT table. 

\par A disadvantage of this design was address space overhead. Testing showed that around 10\% more address space was needed for three simulations on large-scale networks, based on the need to reserve addresses to maintain connections while continuing to change address mappings. However, TAO had the distinct advantage of only requiring the addition of a single box at the network's edge and no cooperation from remote hosts was needed for it to provide its services.

\section{Implementation Proposal}
\label{sec:implementation}
\subsection{Requirements}
\par The Address Routing Gateway focuses on the needs of military networks. These are essentially the needs of any geographically-diverse organization, but we will cover them here to provide a common frame of reference for the goals of ARG. 

\par ARG worries primarily about protecting communication outside of military-controlled networks and preventing external entities from probing internally. This means that our implementation is intended to be employed for all traffic traveling between bases. Base networks, in this case, include both those on permanent installations and those in deployed, forward locations. ARG is not directly concerned with protecting internal networks; that is left to more traditional defenses. However, ARG will have the ability to detect invalid packets and this information can be handed off to those traditional defenses---i.e., an IDS---for analysis.

\par ARG must operate over the commercial Internet. Some proposals for network address space randomization require changes to the existing routing infrastructure \cite{CONTRA}. Deploying such a solution may be possible and beneficial in the long run, but we want a solution that could be deployed today without participation from outside entities.

\par Base networks can generate huge volumes of traffic, so ARG must scale well. It may be possible to divide a particularly large installation into sections, but this may not be feasible or desirable in all situations. Due to the critical nature of military networks, ARG's implementation must not introduce significant latency under any foreseeable load. Likewise, there can never be a brief period where all connections drop. Military networks support critical services and any interruption could be catastrophic. At a minimum, given the number of nodes inside the network, dropped connections would result in a massive amount of wasted bandwidth as they were reestablished and the data retransmitted.

\par Military networks contain a wide range of hardware and software. Much of this software cannot be altered to accommodate ARG, so it must function transparently. Host-level implementations might be feasible for generic workstation images---i.e., an alteration to the operating system's network stack---, but the ability to function in another way must exist to allow legacy equipment to continue operating.

\par Finally, resource constraints---monetary and manpower---preclude some options. ARG is intended to be easily configurable and usable with reasonable hardware investments. While no experimentation has been run yet to determine exact performance, ARG requires only a single gateway system for each network it is intended to protect, rather than hardware and/or software for every client inside those networks.



\chapter{Implementation}
\chapter{Implementation}
\label{chp:implementation}

\par The research this thesis presents relies on a custom \ac{IP} address hopping solution. This chapter covers many of the details of this system, from high-level architecture to the network protocol. Section \ref{sec:arg_requirements} covers the requirements this system strives to meet. Section \ref{sec:arg_impl_overview} gives an architecture overview, while Section \ref{sec:arg_components} covers the details of each component in the system. Section \ref{sec:arg_protocol} details the network protocol used by the system to coordinate gateways.

\section{Requirements}
\label{sec:arg_requirements}
\par \ac{ARG} focuses on the needs of military networks. These are essentially the needs of any geographically-diverse organization: locations throughout the world, high availability and reliability requirements, and security over any network through which its data travels \cite{DialInNetworking}.

\par ARG primarily attempts to protect communication outside of military-controlled networks and prevent external entities from probing internally. This protection includes privacy, as it helps hide information from adversaries by obfuscating sender and receiver information \cite{NetworkBasedPrivacy}. This means that the implementation described is intended to be employed for all traffic traveling between bases, but is unconcerned with internal base traffic. Internal network defense remains the purview of traditional defenses. Base networks, in this case, include both those on permanent installations and those in deployed, forward locations. 

\par ARG must operate over the commercial Internet. Some proposals for network address space randomization require changes to the Internet's existing routing infrastructure and protocols \cite{CONTRA, APOD}. Deploying such a solution may be possible and beneficial in the long run, but a solution that could be deployed today without participation from outside entities is more feasible. This is especially true for forward locations, where traffic is more likely to utilize infrastructure outside the military's control.

\par Like any large, distributed organization, bases can generate huge volumes of traffic, so ARG must scale well. Due to the importance of networks to command and control, ARG's implementation must not introduce significant latency under any foreseeable load. Likewise, there can never be a brief period where all connections drop. At a minimum, given the number of nodes inside the network, dropped connections would result in a massive amount of wasted bandwidth as they were reestablished and the data retransmitted.

\par Military networks contain a wide range of hardware and software. Much of this software cannot be altered to accommodate \ac{ARG}, so it must function transparently. Host-level implementations might be feasible for generic workstation images---i.e., an alteration to the operating system's network stack---, but the ability to function in another way must exist to allow legacy equipment to continue operating.

%\par Finally, resource constraints---monetary and manpower---preclude some options. ARG is intended to be easily configurable and usable with reasonable hardware investments. ARG requires only a single gateway system for each network it is intended to protect, rather than hardware and/or software for every client inside those networks.

\section{Architecture Overview}
\label{sec:arg_impl_overview}
%\par Based on these requirements and the previously cited literature, this paper proposes the Address Routing Gateway. While actual testing of this setup is beyond the scope of this paper, we will attempt to analyze the advantages it would provide, based on previous experimentation.

\par As illustrated in Figure \ref{fig:arg_concept_network}, \ac{ARG} functions entirely around standard networks with hopping gateways. This matches the ``gateway hopping'' scheme discussed in Section \ref{sec:gateway_hopping}. As with \cite{TAO}, these gateways are standalone systems, not intended for use with other tasks. Individual hosts inside these networks have no knowledge of the traffic transformations the gateways perform, whether their connections are routing to a host inside the local network, to a host inside another associated hopping network (hereafter referred to as an ``ARG network''), or to an external network. The implementation of ARG allows the deployment of standard passive defense technologies like firewalls inside the network without reconfiguration. Each gateway maintains a \ac{NAT}-like table to ensure that existing connections are maintained across hops (essentially temporarily leaving the old IP active for just those connections using it already).

\begin{figure}
	\centering
	\includegraphics[width=1\textwidth]{arg_concept_network}
	\caption{\ac{ARG} Conceptual Network Layout}
	\label{fig:arg_concept_network}
\end{figure}

\par Each gateway is given what subnet it is permitted to hop within, a private/public key pair, and the frequency with which it should change \ac{IP} addresses (frequently referred to as its ``hop rate'' or ``hop interval''). For example, \ac{ARG} gateway A in Figure \ref{fig:arg_concept_network} uses subnet 172.100.0.0/16, has the specified public and private keys, and hops every 250 milliseconds. Additionally, each gateway is pre-configured with knowledge of at least one other gateway. The configured information consists solely of the subnet the other gateway is handling and a public key to use for authentication with it. The remaining gateways transfer the remaining information during the connection process. In the figure, this would mean that gateway A knows that gateway B sits in the 172.200.0.0/16 subnet and has the public key XX, but nothing else.

\par At startup, each gateway generates a random symmetric encryption key and a ``hop key.'' They then attempt to connect to all other gateways for which they have configuration files through a series of data and time synchronization packets. Once two gateways fully connect, they exchange data about all other gateways they have configured, allowing ARG nodes to connect to gateways they do not have physical configuration files for. Periodically gateways re-send time synchronization information, ensuring that changing network conditions do not kill communication. Section \ref{sec:arg_protocol} covers this process in more detail.

\par Once connected, packets between ARG networks are encapsulated, encrypted, and authenticated by the originating network's gateway, with each packet given the destination network's current IP. On receipt, a gateway checks that the IPs match what they expect (both its own IP and the source's IP), then validates the signature/\ac{HMAC} before dropping the original packet into their network. \acp{IP} may match either the current \ac{IP} or the previous one, allowing packets sent just before an \ac{IP} change to still be accepted. Packets to external hosts flow through the \ac{NAT}-style system covered in Section \ref{sec:arg_nat}. %These two processes are distinct in their operation and are covered separately in Section \ref{sec:modules}.

%\par Multi-homed networks are not taken into account in this proposal. Further research is needed to see what changes would need to be made to support this setup. If an ARG gateway were placed at each of the connections to the outside network, it is likely that some communication between the two is required to keep them working together \cite{SandiaDynat}, but the exact form this would take needs consideration. \tbd{move to limitations?}

\section{Components}
\label{sec:arg_components}
The handling of packets within ARG is distinctly different if they are to an external (non-ARG network) host or to an ARG network. These processes are handled by two separate components, the hopper and the NAT. A high-level director decides which of these two receives incoming packets. All of these components run as separate threads on the same gateway, closely coordinating their work. Because it is in charge of overall system operation, this section begins by discussing the director.

\subsection{Director}
\label{sec:arg_director}
\par The director is in charge of receiving packets on the internal and external interfaces of the gateway. Upon receipt of a packet, the director parses the packet and decides how to handle it. The decision tree it follows is illustrated in Figure \ref{fig:arg_director_flow} and discussed below.

\begin{figure}
\caption{\ac{ARG} Director Flow}
\label{fig:arg_director_flow}
\centering
\includegraphics[width=1.0\textwidth,clip=true,trim=0 0 0 0]{flow_director}
\end{figure}

\par In the case of \ac{ARP} requests on either interface, the director replies with the gateway's \ac{MAC} address. This feature of \ac{ARG} allows its use without any changes to the network it is placed in, as hosts continue to send to the ``same'' gateway IP as before and \ac{ARG} responds, despite not technically possessing an internal \ac{IP} of its own. For example, the inside hosts send out an \ac{ARP} request for their normal \ac{IP} gateway (as Section \ref{sec:eth_routing} discusses) and the \ac{ARG} gateway sends an \ac{ARP} response, allowing it to grab all traffic and process it appropriately. More details on Ethernet and \ac{IP} routing can be found in Sections \ref{sec:eth_routing} and \ref{sec:ip_routing}.

\par For outgoing packets---packets that are sent from within the protected network that are intended to leave the network---, the director checks the destination \ac{IP} of the packet. If the IP is unknown, it is passed to the \ac{NAT} module. If the IP is within another \ac{ARG} network the director knows of, the packet is handed off to the hopper module to be wrapped and transmitted.

\par For incoming packets---packets hitting the external interface---, the director first checks if the source IP is another ARG network. If it is \textit{not}, the director quickly hands the packet off to the NAT inbound handler.

\par If the packet \textit{is} from another ARG network, the director checks if the packet type indicates it is an administrative packet and hands it off to the hopper's administrative processor. If it is not an administrative packet, then the director confirms that the gateway the packet is from is actually connected. (Which gateway is determined by matching the source \ac{IP} to the base \ac{IP} and mask in the local gateway's configuration.) Assuming that the gateway is connected, the local gateway checks that the source and destination \acp{IP} are correct, based on the data the hopper has on the other gateway. Assuming all those checks pass, the packet is handed off to the hopper to be unwrapped and dropped into the network. If any fail, the packet is silently dropped.

\subsection{Hopper}
\label{sec:arg_hopper}
\par The hopping module is the heart of ARG. It handles maintaining the state of the gateways it knows about---keys, hop rates, times---and using that state to transfer packets to and from the network it is protecting. The other two components (director and \ac{NAT}) talk to the hopper to obtain current IP information and if a given gateway is connected or not. 

\par When the hopper first starts, it initializes a list of gateways with data from configuration files. The information maintained in this structure is shown in Table \ref{tab:gatestate}. 

\begin{table}
\caption{Information Hopper Component Maintains on Other \ac{ARG} Gateways}
\label{tab:gatestate}
\centering
\begin{tabular}{l|l}
	\textbf{Data} & \textbf{Notes } \\
	\hline
	IP Range (IP and mask) & Configuration file \\
	\ac{RSA} Public Key & Configuration file \\
	Hop Interval & Transferred \\
	Symmetric Key & Transferred \\
	Hop Key & Transferred \\
	Time Base & Calculated \\
\end{tabular}
\end{table}

\par An administrative thread is started at the same time. This thread attempts to connect to each gateway in its list periodically and, if it does not hear from a given gateway for several minutes, marks gateways as disconnected. In addition, it sends time synchronization requests fairly frequently to connected gateways, especially if it sees a large percentage of packets being rejected by \ac{IP} address. 

\par Beyond the administrative thread, actions occur in the hopper only when the director passes off packets to it. Outgoing packets are always wrapped, encrypted, and signed, as covered under the ``route packet'' process in Section \ref{sec:arg_protocol_route}. Incoming packets go through the validation process shown in Figure \ref{fig:arg_hopper_in_validation} before being handled. Note that \ac{IP} checking is done in the Director before control reaches the Hopper. After validation exact handling depends on the packet type, but is generally covered in Section \ref{sec:arg_protocol} as part of the protocol discussion.

\begin{figure}
\caption{\ac{ARG} Incoming Packet Validation Process}
\label{fig:arg_hopper_in_validation}
\centering
\includegraphics[width=1.0\textwidth,clip=true,trim=0 0 0 0]{flow_packet_validation}
\end{figure}

\subsection{Network Address Translator}
\label{sec:arg_nat}
\par The \ac{NAT} component of \ac{ARG} maintains a list of on-going connections to external hosts. For instance, if a client inside a protected network connects to 54.24.234.63, the \ac{NAT} creates an entry in an internal table and allows packets from the external host back in to the network and to the client. This is almost identical to the operation of the basic \ac{NAT} discussed in Section \ref{sec:nat}.

\par The only difference from a normal \ac{NAT} system is the addition of an extra field for the current \ac{IP} in the \ac{NAT} table. The new version of the table with example data is shown in Table \ref{tab:arg_nat_example} with the new column in bold.

\begin{table}
\caption{\ac{ARG} \ac{NAT} Table Example}
\label{tab:arg_nat_example}
\centering
\begin{tabular}{r|cccccc}
  & \textbf{Int IP}  & \textbf{Int Port}  & \textbf{Remote IP}  & \textbf{Remote Port}  & \textbf{\textit{Ext IP}}  & \textbf{Ext Port} \\
\hline
1 & \texttt{192.168.0.103} & 3547 & \texttt{74.125.225.69} & 443 & \textbf{\texttt{172.1.123.35}} & 50003\\
2 & \texttt{192.168.0.103} & 8751 & \texttt{207.109.73.34} & 80 & \textbf{\texttt{172.1.73.1}} & 42630\\
3 & \texttt{192.168.0.112} & 30452 & \texttt{4.27.2.253} & 80 & \textbf{\texttt{172.1.86.173}} & 53920
\end{tabular}
\end{table}

\par The traditional \ac{NAT} processing and logic is supplemented with this additional information. When a packet goes out, the table is checked and the packet has its source IP and port changed to the external values. If this is the first packet in a connection, the external IP is filled from the current IP of the gateway. When an incoming packet is encountered, the external IP and port are both checked to determine the correct internal host. The addition of the external IP to the table allows connections to survive across hops; many connections last longer than \ac{ARG}'s intended hop rate and severing connections frequently is unacceptable for many applications. 

\section{Protocol}
\label{sec:arg_protocol}
\par The \ac{ARG} protocol is designed to be fairly stateless, simplifying the implementation and lowering the likelihood of an exploit forcing the gateway into an unexpected state. Packets are sent at the transport layer because port numbers are not needed, but the protocol could be adapted easily to work as a \ac{UDP} payload. \ac{ARG} packets are identified by \ac{IP} protocol 253, a protocol reserved for experimentation \cite{rfc3692}. The header structure of an \ac{ARG} packet is shown in Table \ref{tab:arg_packet_structure}.

\begin{table}
\caption{\ac{ARG} Packet Data}
\label{tab:arg_packet_structure}
\centering
\begin{tabular}{l|l|l}
\textbf{Data} & \textbf{Size} & \textbf{Data Type}\\
\hline
Version & 1 byte & Integer\\
Message Type & 1 byte & Enum, see Table \ref{tbl:arg_protocol_types}\\
Message Length & 2 bytes & Integer\\
Sequence Number & 4 bytes & Integer\\
Signature & 128 bytes & Raw\\
Additional data & 0-32,629 bytes & See Appendix \ref{chp:protocol}
\end{tabular}
\end{table}

\par For this research, the protocol version field is set to 1 at all times. The type field tells the receiving gateway how to process the data contained in the message. Possible values are shown in Table \ref{tbl:arg_protocol_types}. More details on the format of each message type are given in Appendix \ref{chp:protocol}. Length is the network-order size in bytes of the data from the version to the end of the data; given the size of the message header, the minimum for this is 136. The sequence number is a monotonically increasing unsigned integer value used to prevent replay attacks.

\begin{table}
\caption{\ac{ARG} Message Types}
\label{tbl:arg_protocol_types}
\begin{tabular}{l|c|l}
\textbf{Mnemonic} & \textbf{Value} & \textbf{Description}\\
\hline
\texttt{WRAPPED} & 1 & Encapsulated packet from protected client to protected client\\
\texttt{PING} & 2 & Time synchronization message\\
\texttt{CONN\_RESP} & 3 & Connection data response message\\
\texttt{CONN\_REQ} & 4 & Connection data and request for other gateway's data\\ 
\texttt{TRUST\_DATA} & 5 & Configuration information 
\end{tabular}
\end{table}

\par The signature field may actually contain two possible values: a true \ac{RSA} digital signature of the packet or an \ac{HMAC} of the packet, depending on the message type. Packets of type \texttt{PING} or \texttt{CONN\_REQ}/\texttt{CONN\_RESP} are encrypted with the public key of the receiver and signed with the private key of the sender.\footnote{This order is backwards according to best-practice \ac{RSA} and should be corrected to sign-then-encrypt \cite{RobustPrinciplesPK}. It may also be wise to add the name of the sending gateway to each message to help prevent similar mistakes in the future \cite{EngPricCrypto}. The order \ac{ARG} uses allows an attacker to sign a message and claim it as their own \cite{rfc2633}. However, \ac{ARG} gateways will still reject these falsified messages because they will not have a public key for the signer. A gateway could successfully sign another gateway's message, but this indicates a compromised of the gateway itself, at which point the attacker has full access to the entire network anyway.} All other packets---types \texttt{WRAPPED} and \texttt{TRUST\_DATA}---are encrypted with the symmetric key of the receiver and include an \ac{HMAC} of the encrypted data using with the symmetric key of the sender, following standard Encrypt-then-MAC practice \cite{AuthEncryptThenMAC}. The encryption and signing combinations used for each message type as well as whether or not the source and destination \ac{IP} addresses are strictly checked are summarized in Table \ref{tbl:arg_protocol_security}.

\begin{table}
\caption{\ac{ARG} Message Security Summary}
\label{tbl:arg_protocol_security}
\centering
\begin{tabular}{l|l|l|l}
\textbf{Type} & \textbf{Encryption} & \textbf{Signing} & \textbf{\acp{IP} checked?}\\
\hline
\texttt{WRAPPED} & AES, remote key & HMAC, local key & Yes\\
\texttt{PING} & RSA, remote public key & Signature, local private key & No\\
\texttt{CONN\_RESP} & RSA, remote public key & Signature, local private key & No\\
\texttt{CONN\_REQ} & RSA, remote public key & Signature, local private key & No\\
\texttt{TRUST\_DATA} & AES, remote key & HMAC, local key & No\\
\end{tabular}
\end{table}

\par There are four basic exchanges that happen between \ac{ARG} gateways: connect, time sync, trust data exchange, and packet transfer. In order for gateways to begin exchanging packets between the networks they are protecting (via the ``route packet'' process), they must first fully connect by completing the connect and time sync processes. The trust data exchange step is optional, although it allows gateways to connect to others without configuration files. The precise requests, receives, and verifications for each exchange are given in Appendix \ref{chp:protocol}.



\chapter{Methodology}
\chapter{Methodology}
\label{chp:methodology}

\par This chapter discusses the methodology used to measure the effectiveness of \ac{ARG} at correctly classifying valid and invalid traffic, the maximum supportable hop rate at various network latencies, the maximum packet rate \ac{ARG} can handle, and the overall stability of the system under test. Section \ref{sec:problem_def} discusses the problem this research seeks to answer. Section \ref{sec:boundaries} defines the \ac{SUT}, and Section \ref{sec:services} goes into detail on the possible outcomes of the \ac{CUT}. Section \ref{sec:workload} covers the workload presented to the \ac{SUT}, Section \ref{sec:metrics} covers the metrics collected, and Section \ref{sec:parameters} covers the configurable parameters of the \ac{SUT}. Section \ref{sec:factors} brings the previous sections together, giving a comprehensive list of the factors varied for each test. Sections \ref{sec:eval_technique} and \ref{sec:exp_design} detail the actual tests and the purpose of each.

\section{Problem Definition}
\label{sec:problem_def}
\subsection{Goals and Hypothesis}
\label{sec:goals}
\par This research seeks to test whether network address space randomization as discussed in Chapter \ref{chp:implementation} is suitable for deployment on a corporate or military network. Tests against this system are designed to answer four basic questions:

\begin{enumerate}
\item Does \ac{ARG} classify traffic correctly? What percentage of false positives (valid packets blocked) and false negatives (invalid traffic allowed through) does it introduce?
\item What is the maximum packet rate and throughput \ac{ARG} can support?
\item What is the maximum hop rate---the frequency with which \ac{ARG} changes \ac{IP} addresses---that still allows for reliable communication? How does latency affect this?
\item Is \ac{ARG} stable when presented with corrupt, malformed, or replayed packets?
\end{enumerate}

%\par The software developed and tested here attempts to interfere minimally with the network, a critical requirement for the real-world deployment of such technology. Systems \ac{ARG} touches---both inside and outside the ``protected'' networks---do not need any modifications to continue to function. The research done here provides data on whether this is true as a side effect, potentially valuable information for an organization considering employing a \ac{DYNAT} solution. However, validation of this design goal is not a primary objective. 

%\par The working hypothesis for this research is, as speculated by Sandia, a \ac{DYNAT} system allows for quick identification of unexpected (and potentially malicious) packets entering a network. There are few identifiers within the scope of \ac{DYNAT} by which outgoing packets could be filtered. Given that, no filtering is done for outbound packets and hence no change in behavior is observed when compared to the control network. 

\par It is hypothesized that \ac{ARG} correctly classifies 99\% of traffic it encounters when operating with a hop rate appropriate for the network latency. In addition, this thesis hypothesizes that packet loss becomes acceptable when the hop rate matches or exceeds the network latency, where acceptable loss is defined as less than 2\%. This percentage is based on the loss seen on Massachusetts Institute of Technology's wireless networks \cite{MITWifiLoss}. The other two questions under test are informational in nature as the results apply only to this specific hopping gateway implementation, but it is believed that \ac{ARG} is stable in the face of malformed traffic and it can handle at least 10 \ac{Mbps} of traffic. 

\subsection{Approach}
\label{sec:approach}
\par This research is accomplished on a test network with nodes representing the types of hosts found on a typical, corporate-style network. These include trusted hosts inside trusted networks which communicate freely, internal and external servers that must be accessible to hosts inside these trusted networks, and malicious hosts outside the networks. A configurable custom hopping gateway sits in front of the trusted networks. 

\par Traffic generators and collectors run on the test network, determining which traffic flows successfully make it to their intended destination. This includes examining both false positive and false negative rates, determining why \ac{ARG} rejects packets that should get through and why it allows packets that should be rejected. After a given test, logs and traffic captures are collated to form a complete picture of the traffic on the network before determining statistics.

\FloatBarrier
\section{System Boundaries}
\label{sec:boundaries}
\par The \ac{SUT} is \ac{ARG}, the custom \ac{IP} hopping gateway developed specifically for this effort. The basic components of this system, the various inputs into the system, possible outputs, and the metrics provided are illustrated in Figure \ref{fig:sut}. The sections following cover aspects of this diagram in more detail, with Section \ref{sec:services} discussing the possible outcomes, Section \ref{sec:workload} covering the workload, Section \ref{sec:parameters} detailing the parameters in use, and Section \ref{sec:metrics} covering the metrics collected.

\begin{figure}
\caption{\ac{ARG} \ac{SUT} diagram \tbd{ban this sick filth}}
\label{fig:sut}
\centering
\noindent\makebox[\textwidth]{%
\includegraphics[width=1.2\textwidth]{sut}
}
\end{figure}

\FloatBarrier
\section{System Services}
\label{sec:services}
\par This thesis tests three components of \ac{ARG}. Most important is the hopper module, which provides a rapidly-changing external \ac{IP} address and details on connected ARG networks. Using this information, it transports packets between ARG-protected networks. Packets to and from external hosts---hosts that are not part of an ARG network---go through the \ac{NAT} module. Finally, the director module hands packets off to each of the other modules and collects the results back to be logged and potentially acted upon. More details on these components are in Chapter \ref{chp:implementation}.

\par The potential outcomes of the director are shown below, broken into separate sections based on incoming or outgoing packets. Other services do not directly offer outcomes relevant to this research.

\begin{itemize}
\item Director - Incoming
	\begin{itemize}
	\item Accepted: Rewritten and forwarded - Packet is from non-\ac{ARG} network and is rewritten via \ac{NAT} table before forwarding.
	\item Accepted: Unwrapped and forwarded - Packet is from \ac{ARG} network and passes validation checks. Contents are extracted and forwarded internally.

	\item Rejected: Incorrect source \ac{IP} - Packet is coming from an \ac{ARG} network but does not have what the local gateway believes is the current source IP for the other gateway.
	\item Rejected: Incorrect destination \ac{IP} - Packet is coming from an \ac{ARG} network but does not have the current local gateway \ac{IP} as the destination.
	\item Rejected: Incorrect message size - The message length does not match the message type.
	\item Rejected: Incorrect sequence number - The message's sequence number is not monotonically increasing. 
	\item Rejected: Unable to verify signature/\ac{HMAC} - Packet signature invalid/nonexistent (if coming from an \ac{ARG} network).
	\item Rejected: No \ac{NAT} bucket/entry - Packet is coming from a non-\ac{ARG} network but does not have a valid entry in the \ac{NAT} table.

	\item Rejected: Misc - Some operating system-level errors may occur, resulting in rare errors in sending or receiving packets.
	\end{itemize}

\item Director - Outgoing
	\begin{itemize}
	\item Accepted: Rewritten and forwarded - Packet is destined for non-\ac{ARG} network. An entry is made/retrieved from the \ac{NAT} table and used to rewrite packet.

	\item Rejected: Gateway not connected - Packet was intended for an ARG network the gateway is aware of but not yet connected to.
	\item Rejected: Wrapped and forwarded - Packet is destined for an \ac{ARG} network. Wrapped and placed on the external network.

	\item Rejected: Misc - Some operating system-level errors may occur, resulting in rare errors in sending or receiving packets.
	\end{itemize}
\end{itemize}

\section{Workload}
\label{sec:workload}
\par Workload to the system is the traffic flowing through the \ac{ARG} gateways. Standard network traffic parameters like packet rate, packet size, number of simultaneous ongoing connections, and lifetime of connections play a role. However, it is important to note that the network performance itself is not a large concern of this research. Packet rate, for example, does provide useful information about the performance of \ac{ARG} itself, but the numbers apply only to this specific implementation. \ac{ARG}'s development does not focus on performance in this first iteration, so there are many possible areas for improvement. Previous research has shown that similar solutions have minimal impact on performance \cite{NAH}. 

\par Two parameters are more specific to \ac{ARG}. First, the proportion of traffic between protected networks verses traffic to external hosts varies the validation methods \ac{ARG} uses for each packet. Traffic flows that focus on connections between protected networks rely on current \ac{IP} synchronization and signatures, while traffic that originates externally exercises the \ac{NAT} table.

\par Second, the proportion of valid and invalid traffic---traffic that should or should not be permitted through \ac{ARG}---is a workload parameter. The reasons behind each packet's invalidity is also important: most of the possible outcomes from \ac{ARG} depend on \textit{why} a packet is invalid. The possible failure points here are incorrect external \acp{IP}, invalid packet signatures, and no entry in the \ac{NAT} table.

\section{System Parameters}
\label{sec:parameters}
\par As a network application, \ac{ARG} is affected by both the machine on which it runs and the network over which it communicates. \ac{ARG}'s local performance is most affected by processor and memory speeds, with encryption potentially consuming a fair amount of processor time and memory speeds impacting virtually all aspects of operation.

\par The primary physical network parameter that affects \ac{ARG} is latency. To ensure that two \ac{ARG} gateways are able to communicate reliably, packets sent from one gateway to the other must arrive before the \ac{IP} addresses used in the send are no longer current. If hops occur too frequently, then a high one-way latency will cause sent packets to frequently arrive after the receiving gateway has hopped to a different \ac{IP} address. (Adapting to latency is an area of potential improvement, as Section \ref{sec:future_work} discusses.)

\par The primary configuration setting for \ac{ARG} is the hop rate. \ac{ARG} allows the time between hops to be customized from several times a second to minutes apart with millisecond precision. Each gateway may be configured to hop at different rates, but for the sake of this thesis the hop rates are always identical in a given test.

\section{Factors}
\FloatBarrier
\label{sec:factors}
\par Based on the system and workload parameters given above, the following factors are varied as part of the experiment. All others remain constant throughout the experiment. Factors and levels are summarized in Table \ref{tbl:factors} and described in detail below.

\begin{table}
\begin{center}
	\caption{Experimental factors and levels}
	\label{tbl:factors}
	
	\begin{tabular}{r|l}
	\textbf{Factor} & \textbf{Possible Levels} \\
	\hline
	Hop rate (ms) & 1000, 500, 300, 200, 100, 75, 60, 50, 40, 30, 15, 10, 5\\
	Round-trip latency (ms) & 500, 100, 30, 20, 0\\
	Packet delay (s) & 0.3, 0.2, 0.1, 0.05, 0.01, 0.005, 0.001\\
	Traffic direction and type & See detailed description
	\end{tabular}
\end{center}
\end{table}

\begin{itemize}
\item Hop rate
	\par Varying the rate at which \ac{ARG} switches to a new external \ac{IP} allows testing of the maximum supportable hop rate. Hops every 1000 milliseconds give ample time for packets to travel across the network at all but the most extreme latencies, while hops every 5 milliseconds stress even ideal network conditions.

\item Latency between \ac{ARG} gateways
	\par The test network runs through a single switch running four \acp{VLAN}, with an average latency under 1 millisecond. Introducing artificial latency simulates a more realistic range of environments.

\item Packet Delay
	\par Each test spawns a number of traffic generators at different points in the test network. The packet delay shown here is used as a wait between each packet sent. Lower delays result in more frequent sends and force the gateways to deal with more packets per second.

\item Traffic direction and type
	\par \ac{ARG} is capable of handling \ac{TCP}, \ac{UDP}, and \ac{ICMP} traffic. All tests use equal amounts of \ac{TCP} and \ac{UDP} traffic, but some vary the direction. Section \ref{sec:exp_design} gives a full description of the various tests and Figure \ref{fig:testnum_flows} illustrates the direction of all traffic on the network.
\end{itemize}

\section{Evaluation Technique}
\label{sec:eval_technique}
\par Measurement is used to obtain results for each factor level. Due to the fairly complex interactions needed between \ac{ARG} gateways and the processing needed to decide how to handle packets, simulating the system would likely require an equal amount of work with little benefit.

\par Setup of the test environment involves a basic 7-node network: three gateways running \ac{ARG}, one system on the network protected by each gateway, and one host outside the network. Figure \ref{fig:argnetwork} shows the network and the names given to the various systems.

\begin{figure}
	\centering
	\caption{\ac{ARG} test network layout overview}
	\label{fig:argnetwork}
	\includegraphics[width=0.75\textwidth]{thesis_network}
\end{figure}

\par Protected clients behind the gateways (\texttt{ProtA1}, \texttt{ProtB1}, and \texttt{ProtC1}) may communicate freely. The protected clients may also talk out to the external host (\texttt{Ext1}), and the external hosts must then---once that connection is established---be able to talk back into the network. There is additional administrative traffic directly between the gateways (\texttt{GateA}, \texttt{GateB}, and \texttt{GateC}). These three basic traffic flows are ``valid'' traffic.

\par All traffic beyond what is described above is ``invalid.'' For example, \texttt{Ext1} is not allowed to send traffic in to the protected clients or the gateways without them first initiating the connection. Malformed traffic sent by any host is also considered invalid. In either case, invalid traffic should be stopped at the earliest possible opportunity (i.e., the gateway rejects the packet and keeps it from reaching the internal host) and the gateway must remain stable.

\par To collect data, each system runs the traffic collection program \texttt{tcpdump} to capture traffic sent and received into \ac{PCAP} files. The test execution script then spawns traffic generators on the correct systems in the network, based on what test is being run. Section \ref{sec:exp_design} details the types of traffic each test establishes. Each traffic generator logs their sends and receives to a text log file, independent of the \texttt{tcpdump} traffic captures. After a given trial, the \ac{PCAP} files and traffic generator logs are collated and processed with custom scripts to determine the metrics described in Section \ref{sec:metrics}. More details on the traffic generators and test run sequence can be found in Appendixes \ref{chp:testseq} and \ref{chp:generators}. Appendix \ref{chp:processor} covers the custom results processor. \tbd{reread}

\par All trials run on a network of seven physical servers. Each server runs Ubuntu 12.04.1 Server Edition with four gigabytes of \ac{RAM} and a 2.6 gigahertz quad-core Intel Xeon. A single switch with four \acp{VLAN} connects each system at 100 \ac{Mbps}.

\section{Performance Metrics}
\label{sec:metrics}
\par As previously stated, this research primarily focuses on the classification accuracy of \ac{ARG} and the interaction of hop rate and latency. Measurements on \ac{ARG} therefore concentrate on the outcomes from the director. However, basic statistics on network performance are collected. The metrics of interest include:

\begin{itemize}
\item Percentage of invalid packets accepted
	\par If a packet that should have been rejected is accepted by \ac{ARG}, it is possible for an attacker to sneak into the network regardless of the gateway's existence. This is the true measure of whether or not \ac{ARG} is protecting the network. If \ac{ARG} functions correctly, this number should remain at zero for all experiments with \ac{ARG} enabled.

\item Percentage of valid packets rejected
	\par In ideal circumstances, this will also be zero. However, network conditions may result in failures here, which on a real-world network might result in a disruption of service. 

\item Number of each type of rejection (each possible outcome from the director)
	\par This reveals where in the processing stage packets are typically caught. If packets get caught in the later stages of validation---e.g., signature checking---then processing time has been wasted.

\item Packets per second and \acf{Kbps}
	\par An easy check on \ac{ARG}'s performance is comparing the amount of traffic it is handling against the packet loss it shows. Information about both the number of packets and the raw number of bits it processes may reveal slightly different results, so both are collected.
\end{itemize}

\section{Experimental Design}
\label{sec:exp_design}
\par Based on the factors given in Section \ref{sec:factors} and the goals of this research (as presented in Section \ref{sec:goals}), there are eight traffic flows of interest. Each consists of different types of traffic and flow destinations. These are most easily visualized in Figure \ref{fig:testnum_flows}.

\begin{figure}
\caption[Experiment traffic flow directions and protocols]{Experiment traffic flow directions and protocols. Black solid lines indicate valid traffic, red dashed lines are invalid.}
\label{fig:testnum_flows}
\centering
\begin{subfigure}[b]{0.328\linewidth}
	\centering
	\includegraphics[width=1.0\linewidth]{test_traffic_0}
	\caption{Test 0}
\end{subfigure}
\begin{subfigure}[b]{0.328\linewidth}
	\centering
	\includegraphics[width=1.0\linewidth]{test_traffic_1}
	\caption{Test 1}
\end{subfigure}
\begin{subfigure}[b]{0.328\linewidth}
	\centering
	\includegraphics[width=1.0\linewidth]{test_traffic_2}
	\caption{Test 2}
\end{subfigure}
\begin{subfigure}[b]{0.328\linewidth}
	\centering
	\includegraphics[width=1.0\linewidth]{test_traffic_3}
	\caption{Test 3}
\end{subfigure}
\begin{subfigure}[b]{0.328\linewidth}
	\centering
	\includegraphics[width=1.0\linewidth]{test_traffic_4}
	\caption{Test 4}
\end{subfigure}
\begin{subfigure}[b]{0.328\linewidth}
	\centering
	\includegraphics[width=1.0\linewidth]{test_traffic_5}
	\caption{Test 5}
\end{subfigure}
\begin{subfigure}[b]{0.328\linewidth}
	\centering
	\includegraphics[width=1.0\linewidth]{test_traffic_6}
	\caption{Test 6}
\end{subfigure}
\begin{subfigure}[b]{0.328\linewidth}
	\centering
	\includegraphics[width=1.0\linewidth]{test_traffic_7}
	\caption{Test 7}
\end{subfigure}
\begin{subfigure}[b]{0.328\linewidth}
	\centering
	\includegraphics[width=1.0\linewidth]{test_traffic_8}
	\caption{Test 8}
\end{subfigure}
\end{figure}

\par These possible traffic flows are used in four sets of experiments, given below. Each experiment set answers a different research goal.
\begin{itemize}
	\item Basic tests
	\par This sequence of tests verifies that \ac{ARG} classifies traffic correctly by running every test shown in Figure \ref{fig:testnum_flows} against \ac{ARG}. Latency is set to 20 milliseconds (\ac{RTT}) for all tests and traffic generators produce packets around every 0.3 seconds. To determine if hop rate has a statistically significant impact on certain types of traffic, every test runs twice, once with a long hop interval of 500 milliseconds and once with a shorter interval of 50 milliseconds, which is just slightly more than double the round-trip latency.

	\item Max Throughput
	\par This sequence gives an indication of what throughput and packet rate \ac{ARG} is capable of handling. Packet delay goes through all levels shown in Table \ref{tbl:factors}, which leads to roughly corresponding increases in the throughput the gateways must handle. As with the basic tests, the hop interval alternates between 500 ms and 50 ms to see if the additional \ac{IP} calculation load impacts the maximum rate. Test 4 is used across all runs to because it utilizes both \ac{TCP} and \ac{UDP} traffic flowing in all valid directions. \ac{RTT} is set to 20 ms.

	\item Max hop rate
	\par This sequence determines the maximum hop rate at various latencies. Hop rate and latency go through the levels shown in Table \ref{tbl:factors} in a full factorial fashion (every latency-hop rate combination). Packet rate is fixed at 0.3 seconds. Test 4 is used throughout.

	\item Fuzzer
	\par This sequence is not tested rigorously for traffic flow success and failure, but ensures that \ac{ARG} remains stable despite malformed traffic. The traffic flows from Test 8 are used, with additional traffic coming from fuzzers running that replay and/or alter all gateway traffic they see. Hop rate varies between 500 ms and 50 ms, latency is fixed at 20 ms, and packets are sent at 0.3 second intervals.
\end{itemize}

\par A 95\% confidence interval is used for all experiments. Experiments are each run for five minutes, sufficient time for the system to stabilize (pilot studies show that \ac{ARG} fully connects in under 10 seconds on the test network). Wide variation is possible in the actual traffic seen in a single run, so a minimum of 10 replications are used for each experiment.

\section{Summary}
\label{sec:method_summary}
\par This chapter discusses the goals of this research and defines the \ac{SUT} and its relevant factors. The methodology in use is covered, with details on the test network and the exact tests run on this network. Finally, this chapter enumerates the metrics the tests collect and analyze. 



\chapter{Results and Analysis}
\begin{comment}
\lettrine{D}{escribe} the results you obtained when applying the design described in Chapter 3.
At the same time, provide a running analysis of the results. This analysis is probably
the most important part of the whole written thesis, because this is where your real
insights into the problem/solution are documented. Avoid simply showing
plots/tables of results with no analysis.
\end{comment}

\par This chapter presents and analyzes the experimental results. \tbd{sections}

\section{Results and Analysis}


\section{Limitations}
\par The \ac{IP} hopping system this research implements is \tbd{other stuff} \tbd{note ARP triple-latency problem here?}

\section{Summary}



\chapter{Conclusions and Recommendations}
\chapter{Conclusions and Recommendations}
\label{chp:conclusion}
\par This chapter summarizes the work and findings of this research. Section \ref{sec:research_conclusions} summarizes the conclusions reached in this research. Section \ref{sec:research_impact} discusses the impact of this research. Section \ref{sec:future_work} provides recommendations for future work in this area.

\section{Research Conclusions}
\label{sec:research_conclusions}
\par This research has found that \ac{IP} hopping is a suitable method of blocking unexpected external traffic while having a minimal false-positive rate. This can be done in a way completely transparent to the internal and external hosts; the tool developed here works with no configuration changes to other hosts.

\par In addition, \ac{ARG} proves rapid \ac{IP} changes are possible, with network latency as the primary limiter. Tests demonstrate that---under this implemention---\acp{IP} may change around 15 times per second (changes every 50 to 75 milliseconds) and still allow for reliable communication. 

\ac{ARG} also demonstrates good throughput, a critical aspect of deployability to a real network. Test rates reach four \ac{Mbps} with no indication that \ac{ARG} is unable to handle much greater rates. Running a fuzzer against \ac{ARG} found that while gateways themselves remain stable in the face of malformed traffic, it may have an impact on connectivity and valid packet loss.

\section{Research Impact}
\label{sec:research_impact}
\par This thesis presents a new \ac{IP} address hopping tool that combines features of previous efforts in this area. Through a gateway-based solution, \ac{ARG} avoids requiring changes to existing network architecture or any clients inside. \ac{ARG} applies \ac{IP} address changes to all packets entering and leaving the network and packets between \ac{ARG}-protected networks include full encryption and authentication.

\par Of primary importance to this field of research is the demonstration that \ac{IP} changes may occur multiple times per second. Previous research focuses on changes on the participation order of minutes or hours and may kill on-going connections when address changes occur. \ac{ARG}'s design allows for connections to persist across hops without participation of either end of the stream, ultimately allowing for much more frequent address changes and a potential amplification of the benefits of address space randomization.

\section{Future Work}
\label{sec:future_work}
\subsection{IPv6 support}
\par \ac{IPv6} support is slowly becoming an absolute requirement for any network system. For a \ac{IP} hopping system, \ac{IPv6} offers the benefit of a greatly increased address space, allowing systems to hop in a much broader range of addresses. \ac{ARG} is entirely \ac{IPv4} in its operation and cannot transport \ac{IPv6} packets to external hosts or to other gateways.

\subsection{Fragmentation Support}
\par \ac{ARG} currently has no support for fragmenting packets as they pass through the system or of notifying the sender that fragmentation is needed. Packets to and from external hosts pose no problem, as the original sender will handle this themselves. However, packets between gateways/\ac{ARG}-protected networks have additional data added, potentially exceeding the maximum transmission unit of the network. In this case, \ac{ARG} has no way to recover and the packet is permanently dropped without notice. A more complete implementation should notify the sender that fragmentation is needed.

\subsection{More extensive malicious testing}
\par Due to time constraints, a full battery of robust malicious tests could not be performed against \ac{ARG}. As demonstrated by the basic fuzz testing, \ac{ARG} handles errors without dying, but may lose additional packets. The reasons behind this potential issue needs more exploration to determine the root cause and what should be done to fix it. More extensive work in both undirected (i.e., fuzz testing) and directed attacks is needed. For example, malicious hosts might attempt to falsely connect to a gateway or perform replay attacks in a more intelligent manner.  

%\subsection{Red teaming}
%\par In conjunction with the previous suggestion, 

\subsection{More intelligent NAT}
\par \ac{ARG} currently blindly opens holes in the \ac{NAT} when it sees outbound packets and closes them after seeing no activity in a fixed amount of time. A transport layer examination would allow more fine-grained \ac{NAT} work, by watching for actual connection establishment and teardown packets. 

\subsection{Integration with other defenses}
\par Network defenses often perform better when working in tandem. \ac{ARG} has the potential to detect certain types of probes into the network. If this information could be passed off to an \ac{IDS}, it might alert an operator or take other defense actions on the network. In an even more active approach, \ac{ARG} might work with a honeypot to present a fake view of the network to an attacker. By examine what systems an attacker probes, it might be possible to determine the identity of the adversary, their goals, and their intended target in the network, all valuable information to those defending the network.

\section{Summary}
\par This chapter reviews the work and findings of this thesis. The impact of the research is discussed and recommendations for future work are given.



%%%%%%%%%%%%%%%%%%%%%%%%%%
%
% Appendices
%
%%%%%%%%%%%%%%%%%%%%%%%%%%
\appendix    % Indicates the transition from chapters to appendices.  If there
             % is only one appendix it will not be labeled (can't have an "A"
             % without a "B"...).  The labels will be added automatically if 
             % there are more than one.  To get the counter in sync, a third 
             % "latex" may be required on the file.

\chapter{Stuff}
\lettrine{A}{ppendix} goes here

% The \references command should be used to insert the list of references.
% Assuming one is using BibTeX, this should contain both a \bibliographystyle
% and a \bibliography command referencing a separate bibliography file.

\tbd{Need to have the labels as author-year, ie AnA06}
\references{
  \bibliographystyle{thesnumb3} % or ieeetr, spie, aiaa, etc.
  \bibliography{research,rfc}
 }
\end{document}

