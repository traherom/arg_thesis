\chapter{Introduction}
\label{chp:introduction}

\section{Motivation}
\par Traditional network defenses consist of largely static tools; firewalls and intrusion detection systems (IDS) form the backbone of protection in most information technology shops. Despite extensive work and research into these systems, attackers still routinely break into networks and bring down critical systems, exfiltrate data, and establish footholds for future actions. In an effort to combat this, interest has increased in more active defense mechanisms, such as reputation and trust-based security \cite{Untrustworthiness} and \ac{NASR} \cite{APOD, NAH}. This thesis focuses on the latter in a military setting.

\par At a high level, the concept of \ac{NASR} is simple: rather than a system sitting on a single \ac{IP} address, it changes addresses rapidly, hopping amongst a set of \ac{IP} addresses assigned to it. Normally an attacker wishing to target a given network is capable of gaining intelligence through simple scanning, checking each IP inside the network and then checking each port on the active IPs to see what services are available. With this knowledge, the attacker can often find an entrance into the network. IP hopping mitigates this issue by making it difficult to probe the network in the first place and quickly invalidating any network map that the attacker does manage to generate; even if they do manage to look a system's internal IP at one point in time, the address will change just moments later.

\section{Goals and Limitations}
\par This thesis proposes an IP address hopping system called \ac{ARG}. It incorporates many of the features of previous address-hopping schemes, with an eye on the specific needs of the military. In this context each of the existing systems presents drawbacks that \ac{ARG} attempts to avoid. Additionally, the design of ARG is intended to allow its future integration with a traditional IDS and honeypot, potentially gaining additional insight into an attacker's behavior.

\par The geographically diverse military network demands high availablity and reliability, security over any outside network, and the ability to utilize the commercial Internet when necessary for transport. In light of these and other requirements, this thesis examines several questions with regards to \ac{IP} hopping and \ac{ARG}.
\begin{itemize}
	\item Does \ac{ARG} classify traffic correctly? What percentage of false positives (valid packets blocked) and false negatives (invalid traffic allowed through) does it introduce?
	\item What is the maximum packet rate \ac{ARG} can handle?
	\item What is the maximum hop rate---the frequency with which \ac{ARG} changes \acp{IP}---that is supportable? How does latency affect this?
	\item Is \ac{ARG} stable when presented with corrupt, malformed, and/or replayed packets?
\end{itemize}

\par The limitations on the answers to each of these questions varies. If \ac{ARG} shows an acceptable level of classification accuracy, it will demonstrate the viability of (in this regard) of similar systems. Likewise, the maximum supporable hop rate from \ac{ARG} demonstrates that at least that rate is possible. On the other hand, if \ac{ARG} does not support a fast packet rate it only reveals flaws in its own design, not of others. \tbd{not sure about this paragraph. Needed? Leave until analysis?}

%\par \ac{ARG} is a prototype system. While every effort has been put in to implement a protocol suitable for real-world use, \tbd{do we want this?}

\section{Thesis Overview}
\par This chapter introduces the research, goals, and limitations of the work in this thesis. Chapter \ref{chp:background} covers foundational background topics, concepts, and research. Chapter \ref{chp:implementation} discusses the design of \ac{ARG}, including design requirements, architecture, and the network protocol. Chapter \ref{chp:methodology} walks through the test methodology used in this thesis, while Chapter \ref{chp:results} presents results and analysis of these tests. Finally, Chapter \ref{chp:conclusion} provides a concluding discussion of this research and presents possible areas of work in the future.

