\chapter{Traffic Generators}
\label{chp:generators}

\par This appendix covers the required environment for the two traffic generators and their usage. Section \ref{sec:gen_env} lists the required utilities to run the generators. Section \ref{sec:gen_usage} discusses usage of each generator and the available command line parameters.

\section{Environment}
\label{sec:gen_env}
\par The traffic generators run on Ubuntu 12.04, Ubuntu 12.10, and Mac OSX 10.8. Other versions and distributions are untested, although they may work if the below requirements are met. 

\par The following packages must be available for \texttt{gen\_traffic.py} and \texttt{malicious\_traffic.py} to run:

{\singlespace
\begin{itemize}
\item Python 2.7
\item python-scapy $>=$2.2.0
%\item python-libpcap $>=$0.6.2
\end{itemize}
}

\par In Ubuntu:
\begin{lstlisting}[language=bash]
$ sudo apt-get install python-libpcap python-scapy
\end{lstlisting}

\section{Usage}
\label{sec:gen_usage}
\subsection{Normal Traffic Generator}
\par \texttt{gen\_traffic.py} generates random \ac{TCP} and \ac{UDP} packets with the given characteristics and rates. Table \ref{tbl:gen_traffic_param} shows all the available options and the defaults where applicable. The generator can be stopped by pressing \texttt{Ctrl-c} or sending it \texttt{SIGTERM} via \texttt{kill}.

\subsection{Validation}
\par Validation is done manually on this tool by capturing the generated traffic on both the sending and receiving hosts. Full packet logging may be done by the generator directly (raw packets bytes were dumped), so a simple comparison between the sent traffic, the traffic it believes it generated, and the assigned settings ensure all parts of the tool operate as expected.

\subsubsection{Examples}
\par Send \ac{UDP} traffic to 192.168.1.5:2000 twice a second:
\begin{lstlisting}
$ ./gen_traffic.py -t udp -p 2000 -h 192.168.1.5 -d .5
1356728008.79 LOG4 START: Starting at 28 Dec 2012 15:53:28
1356728008.79 LOG4 Starting a valid UDP sender to 192.168.1.5:2000
1356728008.79 LOG4 LOCAL ADDRESS: 192.168.1.115:2000
1356728009.29 LOG4 Sent valid 17:dae8053de4050a230b106d763665f058 to 192.168.1.5:2000
1356728009.29 LOG4 Received valid 17:f537a893140dbcc3b911cb69eae34b46 from 192.168.1.5:2000
1356728009.79 LOG4 Sent valid 17:500b1c56df2e3d17ce50bda4e9be03c9 to 192.168.1.5:2000
...
1356728011.81 LOG4 Sent valid 17:f31dd5564b658e956fe74cc35c0f603f to 192.168.1.5:2000
1356728011.81 LOG4 Received valid 17:ed8104a0ab9835fab38cc453362e8894 from 192.168.1.5:2000
^C1356728012.25 LOG4 User requested we stop
1356728012.25 LOG4 UDP sender to 192.168.1.5:2000 dying
\end{lstlisting}

\par Receive that \ac{UDP} traffic:
\begin{lstlisting}
$ ./gen_traffic.py -t udp -l -p 2000
1356727988.4 LOG4 START: Starting at 28 Dec 2012 15:53:08
1356727988.4 LOG4 Starting a UDP receiver on port 2000
1356727988.4 LOG4 LOCAL ADDRESS: 192.168.1.5:2000
1356728009.29 LOG4 Received valid 17:dae8053de4050a230b106d763665f058 from 192.168.1.115:49958
1356728009.29 LOG4 Sent valid 17:f537a893140dbcc3b911cb69eae34b46 to 192.168.1.115:49958
...
1356728011.81 LOG4 Received valid 17:f31dd5564b658e956fe74cc35c0f603f from 192.168.1.115:49958
1356728011.81 LOG4 Sent valid 17:ed8104a0ab9835fab38cc453362e8894 to 192.168.1.115:49958
^C1356728015.71 LOG4 User requested we stop
1356728015.71 LOG4 UDP listener on port 2000 dying
\end{lstlisting}

\begin{table}
\caption{gen\_traffic.py command-line parameters}
\label{tbl:gen_traffic_param}
\centering
\begin{tabular}{lp{2in}ll}
\toprule
Parameter & Description & Possible Values & Default\\
\hline
\texttt{--type} & Type of traffic to work with & tcp, udp & \\
\texttt{--is-invalid} & Log traffic as invalid. No effect on actual traffic. & & false\\
\texttt{--host} & Host to send to & \ac{IP} or domain name & \\
\texttt{--port} & Port to send to/listen on & Port number 0-65535 & \\
\texttt{--listen} & Listen on given port rather than initiating connection & & false\\
\texttt{--delay} & Delay in seconds between sends & $>=$0.0s & 1.0\\
\texttt{--echo} & Echo received packet data back to sender rather than random & & false\\
\texttt{--size} & Size of packet data to send & Number of bytes to send & Random\\
\texttt{--output} & Log to given file & file name & stdout\\
\bottomrule
\end{tabular}
\end{table}

\subsection{Malicious Traffic Generator}
\par \texttt{malicious\_traffic.py} generates malicious ARG traffic by replaying \ac{ARG} protocol traffic with randomly chosen modifications. The possible modifications are shown below. Each may be chosen with a 10\% probability.

{\singlespace
\begin{itemize}
\item Zero \ac{ARG} signature/\ac{HMAC}
\item Change message type
\item Zero data
\item Remove the data
\item Changed sequence number
\item Changed source \ac{IP} address
\item Changed destination \ac{IP} address
\end{itemize}
}

\par The only option available for \texttt{malicious\_traffic.py} is \texttt{--output} option. This has the same effect as \texttt{gen\_traffic.py}'s and is covered in Table \ref{tbl:gen_traffic_param}.

\subsubsection{Validation}
\par Validation is done manually on this tool by capturing the generated traffic on both the sending and receiving hosts. The generator logs the modification(s) it makes to each packet, so a comparison between the received packet and the replayed packet quickly confirms each modification works as expected. Modifications are checked individually and in combination.

\subsubsection{Limitations}
\par Due to the design of the test network, with a switch connecting the hosts, it is difficult for this tool to reliably sniff traffic (it does not attempt to \ac{ARP} spoof or otherwise redirect traffic flow). To compensate for this the malicious generators are run on the gateways themselves, allowing them to see all of the traffic passing the gateway. A more realistic solution would be to use a true hub (the test network's switch did not allow this configuration) or sniff on a spanning port and send via a separate port.

