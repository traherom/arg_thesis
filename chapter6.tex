\chapter{Conclusions and Recommendations}
\label{chp:conclusion}
\par This chapter summarizes the work and findings of this research. Section \ref{sec:research_conclusions} summarizes the conclusions reached in this research. Section \ref{sec:research_impact} discusses the impact of this research. Section \ref{sec:future_work} provides recommendations for future work in this area.

\par As networks are hit harder each day by increasingly sophisticated attackers, new methods of defending the the systems inside must be developed. An active measure such as IP address hopping has been shown to provide additional protection from network enumeration \cite{NAH}, traffic sniffing \cite{BBNDYNAT}, and exploitation \cite{APOD}.

\par The Address Routing Gateway this paper proposes accomplishes these goals through master gateways that transform traffic without any changes needed to the networks they protect. Previous systems tend to either have a more limited protection scope or require broader changes to existing equipment and networks. While a prototype needs to be implemented to fully demonstrate the system's capabilities, previous work indicates that the overhead associated with the introduced of ARG would fall into an acceptable range \cite{NAH}.

\section{Research Conclusions}
\label{sec:research_conclusions}
\par This research has found that \ac{IP} hopping is a suitable method of blocking unexpected external traffic while having a minimal false-positive rate. This can be done in a way completely transparent to the internal and external hosts; the tool developed here works with no configuration changes to other hosts.

\par In addition, \ac{ARG} proves rapid \ac{IP} changes are possible, with network latency as the primary limiter. \ac{ARG} also demonstrates solid throughput, a critical aspect of deployability to a real network.
\tbd{fuzzer}

\section{Research Impact}
\label{sec:research_impact}
\tbd{sample has this... not in document}

\section{Future Work}
\label{sec:future_work}
\subsection{IPv6 support}
\par \ac{IPv6} support is slowly becoming an absolute requirement for any network system. For a \ac{IP} hopping system, \ac{IPv6} offers the benefit of a greatly increased address space, allowing systems to hop in a much broader range of addresses. \ac{ARG} is entirely \ac{IPv4} in its operation and cannot transport \ac{IPv6} packets to external hosts or to other gateways.

\subsection{Fragmentation Support}
\par \tbd{tbd}

\subsection{More extensive malicous testing}
\par Due to time constraints, a full battery of robust malicious tests could not be performed against \ac{ARG}. As demonstrated by the basic fuzz testing, \ac{ARG} handles errors without dying, but may lose additional packets. The reasons behind this potential issue needs more exploration to determine the root cause and what should be done to fix it. More extensive work in both undirected (i.e., fuzz testing) and directed attacks is needed. For example, malicious hosts might attempt to falsly connect to a gateway or perform replay attacks in a more intelligent manner.  

%\subsection{Red teaming}
%\par In conjunction with the previous suggestion, 

\subsection{More intelligent NAT}
\par \ac{ARG} currently blindly opens holes in the \ac{NAT} when it sees outbound packets and closes them after seeing no activity in a fixed amount of time. A transport layer examination would allow more fine-grained \ac{NAT} work, by watching for actual connection establishment and teardown packets. 

\subsection{Integration with other defenses}
\par Network defenses often perform better when working in tandem. \ac{ARG} has the potential to detect certain types of probes into the network. If this information could be passed off to an \ac{IDS}, it might alert an operator or take other defense actions on the network. In an even more active approach, \ac{ARG} might work with a honeypot to present a fake view of the network to an attacker. By examing what systems an attacker probes, it might be possible to determine the identity of the adversary, their goals, and their intended target in the network, all valuable information to those defending the network.

\section{Summary}
\par This chapter reviews the work and findings of this thesis. The impact of the research is discussed and recommendations for future work are given.

