\lettrine{T}{here} are two distinct parts here:

\par Conclusions: Re-state what you have done, summarize the results that you got,
and summarize the key conclusions that you can draw from your results. All of
this is a summary of the entire thesis, and there should be absolutely nothing new
here that hasn’t already been covered in the previous chapters. (It will probably
be presented in a more concise form, however—don’t just cut-and-paste from
previous chapters).

\par Recommendations: Think of this as a description of what should be done in any
follow-on work (as if a student is picking up where you left off, and you need to
tell them what they could/should look at). I’ve found that a bulleted list of
recommendations often works well, where each bullet is a paragraph (or two)
describing a particular recommendation that you have. Don’t be afraid to point
out the shortcomings of your work, and describe what would need to be done to
overcome these shortcomings. (Doing so shows that you understand the limits of
your research, and comes off far better than trying to pretend that your research
has solved every problem, when it really hasn’t). Much of what is covered here
will have already been stated in the analysis of Chapter 4. However, it’s OK to
have some new concepts here that aren’t explicitly described elsewhere.

\par As networks are hit harder each day by increasingly sophisticated attackers, new methods of defending the the systems inside must be developed. An active measure such as IP address hopping has been shown to provide additional protection from network enumeration \cite{NAH}, traffic sniffing \cite{BBNDYNAT}, and exploitation \cite{APOD}.

\par The Address Routing Gateway this paper proposes accomplishes these goals through master gateways that transform traffic without any changes needed to the networks they protect. Previous systems tend to either have a more limited protection scope or require broader changes to existing equipment and networks. While a prototype needs to be implemented to fully demonstrate the system's capabilities, previous work indicates that the overhead associated with the introduced of ARG would fall into an acceptable range \cite{NAH}.

\par Finally, it is hoped that ARG can be linked with a honeypot and/or intrusion detection system. ARG's unique ability to detect unexpected, potentially malicious packets makes it possible to track the actions of an adversary's actions. This may reveal the identity of an adversary, their goals, and their intended target in the network, all valuable information to those defending the network.

\subsection{Future Work}
\par IPv6 support
\par More extensive fuzzing testing
\par More intelligent NAT setup (ie, tear down connections on RST/FIN)
\par Pass rejection information off to a \ac{IDS} to supplement the information it works with

