\label{chp:conclusion}

\begin{comment}
\lettrine{T}{here} are two distinct parts here:

\par Conclusions: Re-state what you have done, summarize the results that you got,
and summarize the key conclusions that you can draw from your results. All of
this is a summary of the entire thesis, and there should be absolutely nothing new
here that hasn’t already been covered in the previous chapters. (It will probably
be presented in a more concise form, however—don’t just cut-and-paste from
previous chapters).

\par Recommendations: Think of this as a description of what should be done in any
follow-on work (as if a student is picking up where you left off, and you need to
tell them what they could/should look at). I’ve found that a bulleted list of
recommendations often works well, where each bullet is a paragraph (or two)
describing a particular recommendation that you have. Don’t be afraid to point
out the shortcomings of your work, and describe what would need to be done to
overcome these shortcomings. (Doing so shows that you understand the limits of
your research, and comes off far better than trying to pretend that your research
has solved every problem, when it really hasn’t). Much of what is covered here
will have already been stated in the analysis of Chapter 4. However, it’s OK to
have some new concepts here that aren’t explicitly described elsewhere.
\end{comment}

\par As networks are hit harder each day by increasingly sophisticated attackers, new methods of defending the the systems inside must be developed. An active measure such as IP address hopping has been shown to provide additional protection from network enumeration \cite{NAH}, traffic sniffing \cite{BBNDYNAT}, and exploitation \cite{APOD}.

\par The Address Routing Gateway this paper proposes accomplishes these goals through master gateways that transform traffic without any changes needed to the networks they protect. Previous systems tend to either have a more limited protection scope or require broader changes to existing equipment and networks. While a prototype needs to be implemented to fully demonstrate the system's capabilities, previous work indicates that the overhead associated with the introduced of ARG would fall into an acceptable range \cite{NAH}.

\par Finally, it is hoped that ARG can be linked with a honeypot and/or intrusion detection system. ARG's unique ability to detect unexpected, potentially malicious packets makes it possible to track the actions of an adversary's actions. This may reveal the identity of an adversary, their goals, and their intended target in the network, all valuable information to those defending the network.

\tbd{include this stuff?}
\par This research determines if \ac{DYNAT}---the use of a gateway with a rapidly changing external \ac{IP} address---can effectively determine whether traffic should be allowed into a network. Several previous research efforts in this area already proved that \ac{DYNAT} made it more difficult to gain knowledge of a network \cite{BBNDYNAT} and that performance could be minimally impacted \cite{NAH}. 

\par A test network composed of two \ac{DYNAT}-protected networks and a few external hosts is used to explore this question. The custom \ac{DYNAT} solution used here, known as \ac{ARG}, allows the tuning of important system parameters. For the sake of this research, the primary factor is the hop rate. Levels used for experimentation include multiple times a second hops, several times a minute, and no hopping at all. Outside of \ac{ARG} itself, most hosts on the network run traffic generation scripts to simulate network activity. These scripts allow for the adjustment of packet rate, proportion of ``valid'' or ``invalid'' traffic produced, and how much traffic flows between the protected networks verses to the external hosts.

\par All hosts run the packet capture utilities, allowing custom analysis tools to run through each host's logs and pull out the desired metrics. For the research question posed above, the metrics include the overall rejection/acceptance rate of packets, incorrect classification of packets, and why packets are typically rejected. The factors and levels discussed in Section \ref{sec:factors} and the 10 replications result in 810 total experiments.

\section{Research Conclusions}

\section{Research Impact}
\tbd{sample has this... not in document}

\section{Future Work}
\subsection{IPv6 support}
\par \ac{IPv6} support is slowly becoming an absolute requirement for any network system. For a \ac{IP} hopping system, \ac{IPv6} offers the benefit of a greatly increased address space, allowing systems to hop in a much broader range of addresses. \ac{ARG} is entirely \ac{IPv4} in its operation and cannot transport \ac{IPv6} packets to external hosts or to other gateways.

\subsection{More extensive malicous testing}
\par Due to time constraints, a full battery of robust fuzz tests could not be performed against \ac{ARG}. \tbd{As demonstrated by the fuzz tests discussed before... IT DID CRASH. IT DID NOT CRASH. WHO KNOWS}. More extensive work in both undirected (i.e., fuzz testing) and directed attacks is needed. For example, malicious hosts might attempt to falsly connect to a gateway or perform replay attacks in a more intelligent manner. 

\subsection{More intelligent NAT}
\par \ac{ARG} currently blindly opens holes in the \ac{NAT} when it sees outbound packets and closes them after seeing no activity in a fixed amount of time. A transport layer examination would allow more fine-grained \ac{NAT} work, by watching for actual connection establishment and teardown packets. 

\subsection{Integration with other defenses}
\par Network defenses often perform better when working in tandem. \ac{ARG} has the potential to detect certain types of probes into the network. If this information could be passed off to an \ac{IDS}, it might alert an operator or take other defense actions on the network. In an even more active approach, \ac{ARG} might work with a honeypot to present a fake view of the network to an attacker. By examing what systems an attacker probes, it might be possible to determine what the goal of an attack is and who's interests they serve.

\section{Summary}

