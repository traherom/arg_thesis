\chapter{Results Processor}
\label{chp:processor}

\par After every test run, a custom utility processes the pcap and log files into a single database and extracts statistics from there. This appendix documents the usage of this tool and its operation. Section \ref{sec:proc_env} covers the required packages to process runs. Section \ref{sec:proc_usage} covers usage of the results processor. Section \ref{sec:proc_operation} documents the operation of the processor.

\section{Validation}
\par Validation is done manually on this tool via short, extremely low-traffic test runs. By performing tests with only a few packets in each direction, it is possible to manually ensure that all results the processor produces are accurate. These validation tests are done on everything from single-flow tests (one direction between just two hosts) and many-flow, multi-protocol tests.

\section{Environment}
\label{sec:proc_env}
\par The test processor runs on Ubuntu 12.04, Ubuntu 12.10, and Mac OSX 10.8. Other versions and distributions are untested, although they may work if the below requirements are met. 

\par The following packages must be available for \texttt{process\_run.py} to run:

{\singlespace
\begin{itemize}
\item Python 2.7
\item python-scapy $>=$2.2.0
\item python-libpcap $>=$0.6.2
\end{itemize}
}

\par In Ubuntu:
\begin{lstlisting}[language=bash]
$ sudo apt-get install python-libpcap python-scapy
\end{lstlisting}

\section{Usage}
\label{sec:proc_usage}
\par \texttt{process\_run.py} supports a variety of parameters, all of which are optional. The full parameter list and defaults are given in Table \ref{tbl:process_run_param}. Full descriptions of each are given in the list below.

\begin{table}
\caption{\texttt{process\_run.py} command-line parameters}
\label{tbl:process_run_param}
\centering
\begin{tabular}{lll}
\toprule
\textbf{Parameter} & \textbf{Short} & \textbf{Default}\\
\hline
\texttt{--help} & \texttt{-h} & \\
\texttt{--logdir} & \texttt{-l} & `.' \\
\texttt{--database} & \texttt{-db} & In-memory\\
\texttt{--empty-database} & & false\\
\texttt{--skip-processing} & & false\\
\texttt{--skip-stats} & & false\\
\texttt{--offset} & & 0\\
\texttt{--start-offset} & & 0\\
\texttt{--end-offset} & & 0\\
\texttt{--show-cycles} & & false\\
\texttt{--finish-indicator} & & none\\
\bottomrule
\end{tabular}
\end{table}

\begin{itemize}
\item \texttt{--help} - Display usage message.
\item \texttt{--logdir}, \texttt{-l} - Path of directory with log and pcap files.
\item \texttt{--database}, \texttt{-db} - Path to database of processed run data. Will be created if needed, otherwise existing data will be used. If the database is only partially processed, \texttt{process\_run.py} will complete it. If not given, defaults to an entirely in-memory database.
\item \texttt{--empty-database} - If given, all existing data in database is removed.
\item \texttt{--skip-processing} - Do not attempt to process data, only produce results. If the database is incomplete, nothing will happen.
\item \texttt{--skip-stats} - Do not calculate statistics after processing run data.
\item \texttt{--offset} - Number of seconds to ignore at the beginning and end of a run.
\item \texttt{--start-offset} - Number of seconds to ignore at beginning of run. Overrides value of \texttt{--offset}.
\item \texttt{--end-offset} - Number of seconds to ignore at end of run. Overrides value of \texttt{--offset}.
\item \texttt{--show-cycles} - If processing errors occur and the processor generates loops in the packet chains (i.e., following the next hop for each packet eventually would lead back around), show cycles may reveal the problem packet(s). Mostly obsolete.
\item \texttt{--finish-indicator} - File to create after processing is complete. Intended for automation.
\end{itemize}

\subsection{Example}
\par The most common use case, when the caller wants to process the run data in the current directory and create a database named \texttt{run.db} with the results:
\begin{lstlisting}
~/results/basic-t0-l20-hr500ms-2012-11-24-08:54:21$ process_run.py -db run.db
\end{lstlisting}

\par If a run takes several seconds to enter a steady state, it may be beneficial to ignore the first 30 seconds of the run. In addition, this example does its work from a different directory but leaves the results in the same location as the previous example:
\begin{lstlisting}
~/results$ process_run.py -l basic-t0-l20-hr500ms-2012-11-24-08:54:21 -db basic-t0-l20-hr500ms-2012-11-24-08:54:21/run.db --start-offset 30
\end{lstlisting}

\section{Processor Execution}
\label{sec:proc_operation}
\par Run processing follows the steps below. It makes assumptions about the test network to determine where packets are headed and what hosts actually send and receive them, so logs must come from a network set up as documented in Figure \ref{fig:argnetwork}.

\begin{enumerate}
\item Create database.
\item Read run settings from log files and file names.
\item Check settings for test setup problems, such as missing hosts.
\item Read through each \ac{PCAP} file, entering every packet into the database with a hash of its data.
\item Read through each log file. For each send/receive/transformation (gateways passing a packet to/from their inside network) line:
	\begin{enumerate}
	\item Determine the single-hop source and destination of the packet (who sent it and which system should see it next).
	\item Determine the true sender and receiver of packet. That is, who originally sent the packet and for whom it is intended.
	\item Determine if it is intended to be a ``valid'' packet (i.e., should it reach its destination or not?).
	\item Locate the packet in the database via hash and record the log information in the record.
	\item If it is a transformation packet (at a gateway), locate the send and receive packets and link them together.
	\end{enumerate}
\item Trace packets through the network, creating a chain of sends and receives. Packets that pass through the gateway follow the transformation through. That is, if the gateway receives a packet, alters it, and sends it out the other side, the packet sequence continues unbroken.
\item Check for packet cycles, which would indicate a tracing problem.
\item Copy true source and destination of each packet chain into all packets in the sequence. When a host sends a packet, it has an intended recipient. This information is copied into each packet in the chain, making it easier to look up.
\item Locate packet chain terminations. Each packet in a chain is given the ID number of the packet that ends the chain, allowing the processor to tell where each packet ended and if it reached the intended destination.
\item Produce statistics by querying the database for packets matching various criteria, such as packets that terminate at a different destination than intended.
\end{enumerate}

