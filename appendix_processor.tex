\chapter{Results Processor}
\label{chp:processor}

\par After every test run, a custom utility processes the pcap and log files into a single database and extracts statistics from there. This appendix documents the usage of this tool and its operation. Section \ref{sec:proc_env} covers the required packages to process runs. Section \tbd{finish sections}

\section{Environment}
\label{sec:proc_env}
\par The test processor runs on Ubuntu 12.04, Ubuntu 12.10, and Mac OSX 10.8. Other versions and distributions are untested, although they may work if the below requirements are met. 

\par The following packages must be available for \texttt{process\_run.py} to run:
\tbd{versions}
{\singlespace
\begin{itemize}
\item Python 2.7
\item python-scapy $>=$2.2.0
\item python-libpcap
\end{itemize}
}

\par In Ubuntu:
\begin{lstlisting}[language=bash]
sudo apt-get install python-libpcap python-scapy
\end{lstlisting}

\section{Running}

\section{Processor Execution}
\par Run processing follows the steps below:
\begin{enumerate}
\item \tbd{do this list of processor steps}
\end{enumerate}

